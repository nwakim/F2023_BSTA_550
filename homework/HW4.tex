\documentclass[12pt]{article}
%%%%%%%%%%%%%%%%%%%%%%%%%%%%%%%%%%%%%%%%%%%%%%%%%%%%%%%%%%%%%%%%%%%%%%%%%%%%%%%%%%%%%%%%%%%%%%%%%%%%%%%%%%%%%%%%%%%%%%%%%%%%%%%%%%%%%%%%%%%%%%%%%%%%%%%%%%%%%%%%%%%%%%%%%%%%%%%%%%%%%%%%%%%%%%%%%%%%%%%%%%%%%%%%%%%%%%%%%%%%%%%%%%%%%%%%%%%%%%%%%%%%%%%%%%%%
\usepackage{amsmath}
\usepackage{amssymb}
\usepackage{hyperref}
\usepackage{graphicx}
\setcounter{MaxMatrixCols}{10}
%TCIDATA{TCIstyle=LaTeX article (bright).cst}

%TCIDATA{OutputFilter=LATEX.DLL}
%TCIDATA{Version=5.50.0.2890}
%TCIDATA{<META NAME="SaveForMode" CONTENT="1">}
%TCIDATA{BibliographyScheme=Manual}
%TCIDATA{Created=Thursday, November 10, 2005 15:20:28}
%TCIDATA{LastRevised=Monday, August 31, 2009 17:44:14}
%TCIDATA{<META NAME="GraphicsSave" CONTENT="32">}
%TCIDATA{<META NAME="DocumentShell" CONTENT="Exams and Syllabi\SW\Assignment">}

\setlength{\topmargin}{-1.0in}
\setlength{\textheight}{9.25in}
\setlength{\oddsidemargin}{0.0in}
\setlength{\evensidemargin}{0.0in}
\setlength{\textwidth}{6.5in}
\def\labelenumi{\arabic{enumi}.}
\def\theenumi{\arabic{enumi}}
\def\labelenumii{(\alph{enumii})}
\def\theenumii{\alph{enumii}}
\def\p@enumii{\theenumi.}
\def\labelenumiii{\arabic{enumiii}.}
\def\theenumiii{\arabic{enumiii}}
\def\p@enumiii{(\theenumi)(\theenumii)}
\def\labelenumiv{\arabic{enumiv}.}
\def\theenumiv{\arabic{enumiv}}
\def\p@enumiv{\p@enumiii.\theenumiii}
\pagestyle{plain}
\setcounter{secnumdepth}{0}
\parindent=0pt
%\input{tcilatex}
\begin{document}


\begin{center}

Homework \#4

BSTA 550

Due: Saturday, October 23, 2021

%\bigskip
%
\bigskip


% \label{HW1_odd_answers}\textbf{The answers to odd exercises are in the back of the book} 
%\newline \textbf{starting on pg. 621.}

\end{center}


\bigskip

Complete all of the problems listed below. \newline 
Only turn in the ones listed in the ``Turn In" column. \newline
%Please submit problems in the order they are listed and please staple your assignment.
Please submit problems in the order they are listed.

\bigskip

%\textit{You must show all of your work to receive credit.}  \newline 

% Starting with Chapter 10:
\textit{You must show all of your work to receive credit. Don't forget to define every r.v. you use!}  \newline 
\textit{In particular, \textbf{if a similar problem was done in class or an example in the book, make sure to still show every step in the solution and not just cite the examples' results}.}  \newline 



%-----------------------------------
% Assignment          \label{Assignment}
%-----------------------------------

\begin{center}
\begin{tabular}{|c|c||c|}
\hline
Chapter & Turn In & Extra Problems\\
\hline

%1 &     & \# 3, 7, 9, 11 \\  % removed 12
%\hline
%2  & \# 30, NTB \# \ref{NTB_Venn} & \# 1, 4, 8, 16, 19, 23\\  % updated 3 part Venn diagram
%\hline	
%22* & \# 42, NTB \# \ref{NTB_22_Wahrscheinlichkeitstheorie} & \# 3, 5, 7, 25, 27, 30, 31, 39-41, 43-48\\ %removed 3; added 5, 7, 30  ; not collecting 42
%\hline


%3 & \# 10, NTB  \# \ref{Ch_3_pens}, \ref{Ch3_mutindep}   & \#  4, 9, 12, 13 %removed 13 F19; find n with ln()
%\\
%\hline
%4  & \# 12 & \# 1, 4, 11, 13\\
%\hline	
%
%5 & \# 17, NTB \# \ref{Ch5_Bayes_F16Ex1}, \ref{Ch5_Bayes_F05Ex1_solve_for_n}  & \# 1, 9, 11\\%, and problems \ref{Basil}-\ref{Enterprise} below\\
%% F19: \ref{Ch5_UrnMultiplicationRule}, \ref{Ch5_politics}, ; removed F20 since did in class
%\hline

%7 &    & \#  2, 10, 16, 17, 18\\
%\hline	
%8  & \# 8*, 18* & \# 2, 5, 7, 10\\
%\hline	
%% HW 3 CH 9
%9 & \# 2, 8**, NTB \# \ref{Ch9_Smoking_vs_Education}, \ref{Ch9_Nmbr2_extended}  & 1, 4, 9, 10\\
%% moved #9.10 to extra problems since similar to #9.8
%\hline


10* &  \#  6, 8, NTB \# \ref{Ch10_Ch9_Nmbr2_expected_value} & \#  1, 10, 11, 14, 17\\
\hline	
11**  & NTB \# \ref{Ch11_E_Binomial_Female_Vets}, \ref{Ch11_Ch10_Nmbr8_expected_value_Hypergeo}, Book \# 20   & \# 1, 2, 9***, 17***, 18*** \\  % indicator rv: 12, 13
\hline	

%12 & \# 2, NTB  \#\ref{Ch12_Var_linear_combo} - \ref{Ch12_p_hat} & \# 1, 7, 11, 12, 15, 19, 25, 27\\
%% F19: removed #5* since assignment long; 9a (= E[Geo], which already did in Ch10)
%\hline	
%%R & NTB \# \ref{Rsim} below  & \\ % From HW 3 in F15
%%\hline
%13 (review) &   & \# 3, 4, 5, 6, 8, 9, 10, 17, 25 \\
%
%\hline
%20 &    & \# 2, 3, 4\\
%\hline	
%14  &   & \# 3, 7\\
%\hline	
%15 &  NTBP \ref{SumBinom}   & \# 1, 5, 11, 18, 23, NTBP \ref{ParkinsonsBinom}\\  
%\hline	
%16 & \# 8  & \# 3a-g, 11, 21\\
%\hline	
%17 & \# 12a-c  & \# 3a-g, 6, 9, 11, NTBP \ref{SumNegBinom}\\
%% F19: moved NTBP \ref{SumNegBinom} to extra problems
%\hline	
%19 & \# 6  & \# 1, 18, 19\\   % #6 is new F15 (was #5)
%\hline
%18 & \# 24  & \# 1, 26, 27\\  % #24 is new F15 (was #23)
%\hline
%%Calculus Review &  NTB\# \ref{CalculusReview} (g)-(j)  & NTB\# \ref{CalculusReview} (a)-(f)\\  
%Calculus Review &    & NTB\# \ref{CalculusReview} \\  
%\hline
%% Ch 24 461 F15 changed 4 to 3; added 18 to turn in
%% Ch 24 550 F15 Added 20 to turn in with additional parts and moved 18 to list
%24 &  \# 19, 20*  & \# 2, 3, 7, 17, 18, 22, 23\\  

\hline
%R & NTB \# \ref{Rsim} below  & \\ % From HW 3 in F15
%\hline

\end{tabular}
\end{center}


%-----------------------------------
% Comments \label{Comments}
%-----------------------------------


%-----------------------------------
% Chapter 22          \label{COM_22}
%-----------------------------------
%* Please note the following for Chapter 22:
%\begin{itemize}
%\item See the table on pg. 277, which summarizes some key combinatorics concepts.
%%\item I placed a discrete math book (\emph{Discrete Mathematics}, Epp), on reserve for MTH 461 in the library. If you're struggling with this topic, I encourage you to check it out.
%\item Problems 39-48 are a set that build on one another and more advanced than the other problems.  It'll be much easier to do \#42 after doing 39-41.
%\item I \emph{highly} recommend reading Chapter 23, which is a series of case studies in counting: poker hands and Yahtzee. 
%\end{itemize}

%%-----------------------------------
%% Chapter 3          \label{COM_3}
%%-----------------------------------
%
%See also the handout \textit{Conditional Probability Practice} posted in Week 2 Course Materials on Sakai for more practice. 
%
%\bigskip
%
%%* For \#3.13, to receive full credit you must mathematically solve for the sample size instead of plugging in numbers and guessing.
%* For \#3.13, mathematically solve for the sample size instead of plugging in numbers and guessing.

%%-----------------------------------
%% Chapter 8          \label{COM_8}
%%-----------------------------------
%
%* In addition to the graphs, include piecewise defined functions for the pmf and cdf.
%
%%-----------------------------------
%% Chapter 9          \label{COM_9}
%%-----------------------------------
%
%** Break up your solution to Chapter 9 \#8 into the following 5 parts:
%	\begin{enumerate}
%	\item Make a table of the joint probabilities for $X$ and $Y$.
%	\item Using the table in the previous part, write down the piecewise-defined equation for $p_{X,Y}(x,y)$. There should be only 3 pieces (cases) for $p_{X,Y}(x,y)$.
%	\item Express $p_Y(y)$ as a formula (i.e. a function in terms of $y$). 
%	\item Find the conditional pmf $p_{X|Y}(x|y)$ and express your answer as a piecewise-defined equation. There should be only 3 pieces (cases) for $p_{X|Y}(x|y)$.
%	\item Make a table of the joint cdf $F_{X,Y}(x,y)$ values.
%	\end{enumerate}

%*** You may give the joint pmf and cdf functions as tables instead of piecewise defined functions.

%*** Chapter 9 is being split into two parts over HW 3 (joint distributions) and HW 4 (conditional distributions).

%-----------------------------------
% Chapter 10          \label{COM_10}
%-----------------------------------

* Use Chapter 10 techniques when computing expected values for Chapter 10 problems, i.e. computing the expected value directly using the definition of $\mathbb{E}[X]$.
%-----------------------------------
% Chapter 11          \label{COM_11}
%-----------------------------------

** Use Chapter 11 techniques when computing expected values for Chapter 11 problems, i.e. expressing the r.v. as a sum of other r.v.'s and calculating the expected value of the sum of r.v.'s. \newline
Also, as I mentioned in class and posted on Sakai, we will  be skipping the more complex examples of finding expected values using indicator r.v.'s. You can skip Examples 11.5, 11.10, and 11.11. We will not be covering these techniques. 

*** Although Chapter 11 exercises, these are to be done using Chapter 10 techniques since we aren't covering the more complex examples of finding expected values using indicator r.v.'s. 

%-----------------------------------
% Chapter 12          \label{COM_12}
%-----------------------------------

%\bigskip
%% 12.5 hint:
%* (Ch 12) Hint: See example 12.14 on pg. 149.
%
%\bigskip


%-----------------------------------
% Chapter 24          \label{COM_24}
%-----------------------------------

%** (Ch 24) Also find the cdf $F_X(x)$.


%-----------------------------------
% End Comments
%-----------------------------------

\rule{500pt}{1pt}
\bigskip


 Non-textbook problems (NTB):          \label{NTB}

\begin{enumerate}  % for all NTB

%-----------------------------------
% Chapter 2 Venn Diagram          \label{NTB_2}
%-----------------------------------
%% updated the Facebook, Google, YouTube example
%\item \label{NTB_Venn} Suppose the following are the percentage of US adults with the following conditions:
%	\begin{itemize}
%		\item $A$: Hypertension 33\%
%		\item $B$: Diabetes 9\%
%		\item $C$: Metabolic syndrome 24\%
%		\item $A$ or $B$: 39\%
%		\item $A$ or $C$: 45\%
%		\item $B$ or $C$: 28\%
%		\item $A$ or $B$ or $C$: 48\%
%	\end{itemize}
%\begin{enumerate}
%\item Make a Venn diagram of the 3 conditions labeling the percentage (or probability) for \emph{ALL} of the 8 ``sections".
%\emph{Hint: Start from the last condition and work your way up!}
%\item For each of the following (1. - 7. below), \newline
%($i$) write out the event using unions, intersections, and/or complements of the events $A$, $B$, and $C$ (this is NOT finding the probability, that's in $ii$);  \newline
%($ii$) find the probability of the event; and  \newline
%($iii$) write a sentence explaining what the probability is of in terms of the context of the problem.
%	\begin{enumerate}
%		\item $\mathbb{P}$(event at least one of the 3)
%		\item $\mathbb{P}$(event none)
%		\item $\mathbb{P}$(event $A$ only)
%		\item $\mathbb{P}$(event exactly one)
%		\item $\mathbb{P}$(event $A$ and $B$)
%		\item $\mathbb{P}$(event $A$ and $B$ but not $C$)
%		\item $\mathbb{P}$(event all 3)
%	\end{enumerate}
%
%\end{enumerate}



%-----------------------------------
% Chapter 2 Venn Diagram          \label{NTB_2old}
%-----------------------------------
%\item Suppose the following are the percentage of internet users using the following applications:
%	\begin{itemize}
%		\item A: Facebook 70\%
%		\item B: YouTube 80\%
%		\item C: Google 75\%
%		\item A or B: 85\%
%		\item A or C: 90\%
%		\item B or C: 95\%
%		\item A or B or C: 98\%
%	\end{itemize}
%Make a Venn diagram labeling the percentage (or probability) for \emph{ALL} of the 8 ``sections" and then answer the following questions.
%\emph{Hint: Start from the last condition and work your way up!}
%Find 
%\begin{enumerate}
%	\item P(at least one of the 3)
%	\item P(none)
%	\item P(A only)
%	\item P(exactly one)
%\end{enumerate}
%-----------------------------------



%-----------------------------------
% Chapter 22           \label{NTB_22}
%-----------------------------------


%-----------------------------------
% Chapter 22 - choir
%-----------------------------------

%% From F17 Exam 1; on F18 hw
%\item \label{NTB_22_choir} The director of a choral group decided that at every rehearsal the choir would sit in a different configuration. There are four sections in the choir: basses (B), tenors (T), altos (A), and sopranos (S). Assume that basses and tenors are all men and altos and sopranos are all women.
%The seating at rehearsal has the director in the middle with 2 rows of singers on both sides of him facing towards the center. The diagram below is an example with 4 singers in each of the sections.
%
%\smallskip
%
%B\  \  T\  \   director\  \   A\  \   S  \newline
%B\  \   T\  \  \  \  \  \  \  \  \  \  \  \  \  \  A\  \   S \newline
%B\  \   T\  \  \  \  \  \  \  \  \  \  \  \  \  \  A\  \   S \newline
%B\  \   T\  \  \  \  \  \  \  \  \  \  \  \  \  \  A\  \   S \newline
%
%\begin{enumerate}
%
%\item Assuming there is assigned seating within each section, how many ways are there to arrange the sections for rehearsal?
%
%\item A problem that arose with this seating plan was that one couldn't hear the women singing if men were sitting in front of them, and it was decided that no seating plans with women sitting behind men would be allowed. For example, the following arrangement would not be allowed: \newline
%
%A  \  T\  \   director\  \   B\  \   S  \newline
%A\  \   T\  \  \  \  \  \  \  \  \  \  \  \  \  \  B\  \   S \newline
%...
%
%
%With this restriction, how many ways are there to arrange the sections for rehearsal (assuming there is assigned seating within each section)?
%
%\textbf{For the remaining parts, assume men can sit in front of women (as in part (a))}
%
%\item The choir consists of 10 basses, 9 tenors, 11 altos, and 12 sopranos. How many seating arrangements are possible if singers have to sit together within their respective sections but do not have assigned seating?
%
%\item To complicate matters, there are actually two parts within each section. There are 5 B1's, 5 B2's, 4 T1's, 5 T2's, 5 A1's, 6 A2's, 6 S1's, and 6 S2's. How many ways are there to arrange the singers if they are sitting not only within their sections, but the singers on the $1^{st}$ part are all together and the singers on the $2^{nd}$ part are all together?
%
%\item How many ways are there to arrange the singers if they are sitting within their sections, but members of the same part cannot sit next to each other? For example, a B1 cannot sit next to a B1 and a B2 cannot sit next to a B2. Use the numbers on each part given in (d).
%
%
%\end{enumerate}


%-----------------------------------
% Chapter 22 - Wahrscheinlichkeitstheorie
%-----------------------------------

%% From F18 Exam 1; on F19 hw
%% Similar to L&M(3) 2.10.12, HW 3-5 F05
%\item \label{NTB_22_Wahrscheinlichkeitstheorie} The German word for probability theory is
%$$
%W\ A\ H\ R\ S\ C\ H\ E\ I\ N\ L\ I\ C\ H\ K\ E\ I\ T\ S\ T\ H\ E\ O\ R\ I\ E
%$$
%If the letters in this word are arranged at random, 
%
%\begin{enumerate}
%\item what is the probability that none of the H's will be adjacent?
%
%\item what is the probability that not all of the H's will be adjacent?
%\end{enumerate}


%-----------------------------------
% Chapter 3           \label{R_3}
%-----------------------------------

%\item \emph{R Simulation Question.} Complete this problem using R and R Markdown. Upload to Sakai both your .Rmd file as well as the output thereof (Word, html, or pdf) . Please format your markdown code so that the output file displays both your R code and the corresponding R output.  
%
%\emph{Practice with the $\mathtt{sample}$ function}
%
%Recall that the $\mathtt{sample}$ function in R takes a sample of specified size from the elements given in its first argument.
%
%\begin{enumerate}
%\item Look up the help file for the $\mathtt{sample}$ function. Explain what the argument \emph{replace} does in the $\mathtt{sample}$ function.
%\item Use the $\mathtt{sample}$ function to simulate ten rolls of a six-sided (fair) die. Apply the $\mathtt{table}$ command to get a quick summary of your results.
%\item Suppose you have a population of 1000 people, where 500 are Democrat, 300 are Republican, and 200 are Independent. 
%	\begin{enumerate}
%	\item How many Democrats would you expect to get in a simple random sample of 10 people from this population? \emph{Calculate this exactly using arithmetic, meaning do not simulate it.}
%	\item Use the $\mathtt{sample}$ function to simulate taking a simple random sample of 10 people from this population, where you only record the person's political preference ("D", "R", or "I"). How many Democrats did you get in your sample? How does this compare to your answer in part i.? 
%	\newline \emph{Hint: First you need to create a dataset consisting of D's, R's, and I's to sample from. The $\mathtt{rep}$ command might be useful for this.}
%	\item Repeat the previous step 100 times and save the number of Democrats in each sample. 
%		\begin{enumerate}
%		\item What is the mean number of Democrats from these 100 simulations? Is your answer surprising? Why or why not?
%		\item Make a histogram of the 100 results, and describe the distribution shape. What are the min and max number of Democrats selected from the 100 simulations? 
%		\end{enumerate}
%	\end{enumerate}
%\end{enumerate}


%-----------------------------------
% Chapter 3           \label{NTB_3}
%-----------------------------------


%\item \label{geometric_series} Prove the formula for the sum of a geometric series. You may look this up on the internet if you like. 

%%-----------------------------------
%% Chapter 3- pen colors
%%-----------------------------------
%
%% F18 Exam 1 question
%
%\item \label{Ch_3_pens} Deep in the depths of a student's backpack is a collection of pens in different colors. There are 5 black, 4 blue, 3 green, and  2 purple pens. 
%
%\begin{enumerate}
%\item If the student randomly selects 8 pens from the backpack \underline{without} replacement, what is the probability that they took 2 black, 3 blue, 2 green, and 1 purple pens?
%
%\item If the student randomly selects 8 pens from the backpack \underline{with} replacement (returning the selected pen after each draw), what is the probability that they took a purple pen 3 times?
%
%\item If the student randomly selects pens from the backpack \underline{with} replacement (returning the selected pen after each draw) until they get the $3^{rd}$ green pen, what is the probability that they get the $3^{rd}$ green pen on the $10^{th}$ draw?
%
%% remove: Derive the expected value of the r.v. $X$ using indicator functions
%%\item Let the random variable $X$ denote how many random draws from the backpack \underline{with} replacement (returning the selected pen after each draw) are needed until the first green pen is selected. Derive the expected value of the r.v. $X$ using indicator functions. 
%\end{enumerate}

%%-----------------------------------
%% Chapter 3 Independence of 3 events
%%-----------------------------------
%
%\item \label{Ch3_mutindep} Recall from class, that we defined events $A,B,$ and $C$ to mutually
%independent if both (1) and (2) below hold. This point of this exercise is to
%show that $(1)\nRightarrow (2),$ and $(2)\nRightarrow (1).$%
%\begin{equation*}
%\begin{array}{cc}
%(1) & \mathbb{P}(A\cap B\cap C)=\mathbb{P}(A)\mathbb{P(}B)\mathbb{P(}C) \\ 
%(2) & \mathbb{P}(A\cap B)=\mathbb{P}(A)\mathbb{P(}B) \\ 
%& \mathbb{P}(A\cap C)=\mathbb{P}(A)\mathbb{P(}C) \\ 
%& \mathbb{P}(B\cap C)=\mathbb{P}(B)\mathbb{P(}C)%
%\end{array}%
%\end{equation*}
%
%\begin{enumerate}
%\item \label{} Suppose two different fair dice are rolled. Let events $A,B,$ and $C$
%be defined in the following way:%
%\begin{equation*}
%\begin{array}{cl}
%A: & \text{Roll a total of 7} \\ 
%B: & \text{First die is a 6} \\ 
%C: & \text{Second die is a 2}%
%\end{array}%
%\end{equation*}
%
%Show that condition $(2)\ $holds, but that condition (1) does not.
%
%\item Suppose two different fair dice are rolled. Let events $A,B,$ and $C$
%be defined in the following way:%
%\begin{equation*}
%\begin{array}{cl}
%A: & \text{Roll a 1 or 2 on the first die} \\ 
%B: & \text{Roll a 3, 4, or 5 on the second die} \\ 
%C: & \text{Roll a total of 4, 11, or 12}%
%\end{array}%
%\end{equation*}
%
%Show that condition $(1)\ $holds, but that condition (2) does not.
%\end{enumerate}  % end Chapter 3 Independence of 3 events
%
%%-----------------------------------



%-----------------------------------
% R Simulation Question
%-----------------------------------

%
%\item \label{Rsim} \emph{R Simulation Question.} Complete this problem using R and R Markdown. Upload to Sakai both your .Rmd file as well as the output thereof (Word, html, or pdf) . Please format your markdown code so that the output file displays both your R code and the corresponding R output.  
%
%\emph{Simulating the probability of $r$ heads in $n$ coin tosses} \newline
%\begin{itemize}
%\item For these coin problems, assume that a fair coin is being used.
%\item To be able to discuss your simulation results without worrying that they change each time you compile your file, I recommend first setting a seed in R: $\mathtt{set.seed()}$.
%\end{itemize}
%
%\begin{enumerate}
%\item Using 10,000 trials (simulations), simulate the following probabilities  
%	\begin{enumerate}
%     \item Probability of three heads in three coin tosses.
%     \item Probability of 61 heads in 100 coin tosses.
%     \item Create an R function that will simulate the probability of $r$ heads in $n$ coin tosses with $m$ simulations, where $r$, $n$, and $m$ are parameters the user can choose.
%     \item Compare the answers your function creates with those of parts 1 and 2. Are they similar?
%     \item  Compare your answers of parts 1 and 2 to the actual probabilities found by. Are they similar?
%	\end{enumerate}
%\end{enumerate}
%
%\item Still using 10,000 trials and a fair coin, simulate 100 coin tosses and keep track of the number of heads on each trial.
%	\begin{enumerate}
%	\item Make a histogram of the number of heads in 100 tosses from each of the simulations. Describe the shape of the distribution, center and spread (what are the min and max? ``most" values are typically between ... and ...).
%	\item Compute the mean and standard deviation for the number of heads from the simulations.
%	\item What percent of values (number of heads) were within one standard deviation of the mean? Two sd? Three sd? Do these percentages look familiar to you? If so, from where?
%	\end{enumerate}


%-----------------------------------
% Chapter 4           \label{NTB_4}
%-----------------------------------



%-----------------------------------
% Chapter 5           \label{NTB_5}
%-----------------------------------

%%F06 Exam 1: Law of Total Probability 
%\item \label{Ch5_RatsMaze} In a T-maze, a laboratory rat is given the choice of going to the left and getting food or going to the right and receiving a mild electric shock. Assume that before any conditioning (in trial number 1) rats are equally likely to go the left or to the right. After having received food on a particular trial, the probability of going to the left and right become 0.6 and 0.4, respectively on the following trial. However, after receiving a shock on a particular trial, the probabilities of going to the left and right on the next trial are 0.8 and 0.2, respectively. What is the probability that the animal will turn left on trial number 2?


%%-----------------------------------
%% Chapter 5: from practice handout: General Multiplication Rule         
%%-----------------------------------
%\item \label{Ch5_UrnMultiplicationRule}An urn contains 6 white chips, 4 black chips, and 5 red chips. Five chips are drawn one at a time and without replacement.
%
%\begin{enumerate}
%\item What is the probability of getting the sequence B B R W W?
%\item Suppose now that the chips are numbered 1 through 15. What is the probability of getting a specific sequence - say 2, 6, 4, 9, 13?
%\end{enumerate}

%%-----------------------------------
%% Chapter 5: from practice handout: Law of Total Probability         
%%-----------------------------------
%
%\item \label{Ch5_politics}In an upstate congressional race, the incumbent Republican ($R$) is running against a field of three Democrats ($D_1,D_2,D_3$) seeking the nomination. Political pundits estimate that the probabilities of $D_1,D_2,$ and $D_3$ winning the primary are 0.35, 0.40, and 0.25, respectively. Furthermore, results from a variety of polls are suggesting that $R$ would have a 40\% chance of defeating $D_1$ in the general election, a 35\% chance of defeating $D_2$, and a 60\% chance of defeating $D_3$. Assuming all these estimates to be accurate, what are the chances that the Republican will retain his seat?

%-----------------------------------
% Chapter 5: from practice handout: Law of Total Probability         
%-----------------------------------
%\item \label{Ch5_Enterprise}The crew of the Starship \textit{Enterprise} is considering launching a surprise attack against the Borg in a neutral quadrant. Possible interference by the Klingons, though, is causing Captain Picard and Data to reassess their strategy. According to Data's calculations, the probability of the Klingons joining forces with the Borg is 0.2384. Captain Picard feels that the probability of the attack being successful is 0.8 if the \textit{Enterprise} can catch the Borg alone, but only 0.3 if they have to engage both adversaries. Data claims that the mission would be a tactical misadventure if its probability of success were not at least 0.7306. Should the \textit{Enterprise} attack?

%\emph{Answer: Attacking would be a tactical misadventure (0.6808).}


%-----------------------------------
% Chapter 5: from practice handout: Bayes' Rule         
%-----------------------------------

%\item \label{Ch5_Basil} Brett and Margo have each though about murdering their rich Uncle Basil in hopes of claiming their inheritance a bit early. Hoping to take advantage of Basil's predilection for immoderate desserts, Brett has put rat poison in the cherries flambe; Margo, unaware of Brett's activities, has laced the chocolate mousse with cyanide. Given the amounts likely to be eaten, the probability of the rat poison being fatal is 0.60; the cyanide, 0.90. Based on other dinners where Basil was presented with the same dessert options, we can assume that he has a 50\% chance of asking for the cherries flambe, a 40\% chance of ordering the chocolate mousse, and a 10\% chance of skipping dessert altogether. No sooner are the dishes cleared away when Basil drops dead. In the absence of any other evidence, who should be considered the prime suspect?

%\emph{Answer: Margo should be considered the prime suspect (0.55).}



%%-----------------------------------
%% Chapter 5: Bayes' Rule problem from F16 Exam 1   
%%-----------------------------------
%
%%F16 Exam 1
%% Similar to F12; need a hypergeometric counting argument
%\item  \label{Ch5_Bayes_F16Ex1} A new drug is packaged to contain 30 pills in a bottle. Suppose that 98\% of all bottles contain no defective pills, 1.5\% contain one defective pill, and 0.5\% contain two defective pills. Two pills from a bottle are randomly selected and tested. What is the probability that there are 2 defective pills in the bottle given that one of the two tested pills is defective?
%



%%-----------------------------------
%%2017-11-12: Have not assigned this one yet, but should! It's tricky :)
%% Chapter 5: Bayes' Rule problem from F05 Exam 1 - solve for n; tricky!!!   
%%-----------------------------------
%
%\item \label{Ch5_Bayes_F05Ex1_solve_for_n} Suppose you are taking a multiple choice exam with 20 questions, where each question has five choices for an answer. Some of the questions you know the answer to and the others you guess by choosing one of the five choices randomly. Suppose that the probability of your knowing the correct answer to a randomly selected question given that you got it right is 0.88. How many of the 20 questions were you prepared for?
%
%%\noindent\underline{\textsc{Solution}}:\newline
%%Let $C$ be the event you got a randomly selected questions  correct. Let $P$
%%be the event you were prepared for the question.  Then $\mathop{\rm
%%I\negthinspace P}[C|P]=1$, $\mathop{\rm I\negthinspace P}[C|P^C]=0.20$, $%
%%\mathop{\rm I\negthinspace P}[P]=p$, and  $\mathop{\rm I\negthinspace P}[P^C]%
%%=1-p$. Thus  
%%\begin{eqnarray}
%%\mathop{\rm I\negthinspace P}[P|C]=0.88&=&\frac{\mathop{\rm I\negthinspace
%%P}[C|P]\mathop{\rm I\negthinspace P}[P]}{\mathop{\rm I\negthinspace P}[C|P]%
%%\mathop{\rm I\negthinspace P}[P]+\mathop{\rm I\negthinspace P}[C|P^C]%
%%\mathop{\rm I\negthinspace P}[P^C]}  \nonumber \\
%%&=&\frac{1\cdot p}{1\cdot p+(0.20)(1-p)}  \nonumber
%%\end{eqnarray}
%%Solving this gives you $p=0.5946$, and hence $20(0.5946)=11.89$, or 12,
%%questions were prepared for.




%-----------------------------------
% Chapter 9           \label{NTB_9}
%-----------------------------------

%%Joint and conditional distributions: Smoking vs. Education Level
%\item  \label{Ch9_Smoking_vs_Education} 
%The following table shows the results of a survey in which the subjects were a sample of 300 adults residing in a certain metropolitan area. Each subject was asked to indicate which of three policies they favored with respect to smoking in public places. (Table is from\emph{ Biostatistics: A Foundation for Analysis in the Health Sciences}, 10th Edition, Daniel, Wayne W.; Cross, Chad L., pg. 630)
%
%%\usepackage{graphicx}
%\begin{center}
%  \includegraphics[width=6in]{Table_Smoking_vs_Education_Level.pdf}
%\end{center}
%
%Let $X=$ highest education level and $Y=$ policy favored. We can let $X=1$ for college graduate, $X=2$ for high-school graduate, etc., and similarly for $Y$, or just keep the category names for the different levels of $X$ and $Y$
%	\begin{enumerate}
%     \item Make a table for the joint pmf $p_{X,Y}(x,y)$ and briefly describe in words what the values are the probability of.     
%     \item Find the marginal pmf $p_{X}(x)$  and briefly describe in words what the values are the probability of.     
%     \item Find the marginal pmf $p_{Y}(y)$  and briefly describe in words what the values are the probability of.
%     \item Make a table for the joint cdf $F_{X,Y}(x,y)$  and briefly describe in words what the values are the probability of.
%      \item Find the marginal cdf $F_{X}(x)$  and briefly describe in words what the values are the probability of.     
%     \item Find the marginal cdf $F_{Y}(y)$  and briefly describe in words what the values are the probability of.
%% g-h
%     \item Make a table for the conditional pmf $p_{X|Y}(x|y)$  and briefly describe in words what the values are the probability of.
%     \item Make a table for the conditional pmf $p_{Y|X}(y|x)$  and briefly describe in words what the values are the probability of.
%	\end{enumerate}


%% Ch9 #2 is a joint Geo similar to example from class
%\item  \label{Ch9_Nmbr2_extended} \textbf{Forgetful mornings revisited}.
%Using the joint pmf you found in Chapter 9 \#2, complete the following questions:
%	\begin{enumerate}
%     \item Find the joint cdf of $X$ and $Y$ and briefly explain what $F_{X,Y}(x,y)$ represents in the context of the problem.
%     \item Find the conditional pmf $p_{Y|X}(y|x)$.
%	\end{enumerate}

%-----------------------------------
% Chapter 10           \label{NTB_10}
%-----------------------------------

% Ch9 #2 is a joint Geo similar to example from class
\item  \label{Ch10_Ch9_Nmbr2_expected_value} \textbf{Forgetful mornings revisited again}.
Recall from Chapter 9 \#2, that $X$ is the number of days until Maude loses her cell phone and each day she has a 1\% chance of losing her phone (her behavior on different days being independent). For this problem, ignore the r.v. $Y$, and consider the r.v. $X$ on its own.
	\begin{enumerate}
     \item What is the pmf of $X$?
     \item Use the pmf of $X$ to find $\mathbb{E}[X]$.
	\end{enumerate}


%-----------------------------------
% Chapter 11           \label{NTB_11}
%-----------------------------------

%% Ch9 #2 is a joint Geo similar to example from class
%\item  \label{Ch11_Ch9_Nmbr2_expected_value_Binomials} \textbf{Forgetful mornings revisited again}.
%Recall from Chapter 9 \#2, that Maude has a 3\% chance of forgetting to eat breakfast (her behavior on different days being independent). During a 30 day month, how many days do we expect Maude to forget to eat breakfast? \emph{Don't forget to show all your work!!}


\item  \label{Ch11_E_Binomial_Female_Vets}
Approximately 10\% of U.S. Veterans are women. Suppose an investigator plans a study with 4500 participants that are Veterans. How many women can they expect to be included? \textit{Your answer must be calculated by defining a random variable and showing how to calculate the expected value.}

\item  \label{Ch11_Ch10_Nmbr8_expected_value_Hypergeo} \textbf{Cashews revisited}.
Recall from Chapter 10 \#8, that a bowl contains 30 cashews, 20 pecans, 25 almonds, and 25 walnuts, and 3 nuts are randomly selected to eat (without replacement). Again, find the expected value of the number of cashews, but this time by defining the number of cashews as a sum of random variables. 


%-----------------------------------
% Chapter 12: Mean and variance of linear combinations of r.v.'s          \label{NTB_12}
%-----------------------------------

%\item  \label{Ch12_Var_linear_combo} Prove that for a r.v. $X$ and constants $a$ and $b$, that $$\mathrm{Var}[aX+b]=a^2\mathrm{Var}[X].$$ Note: you will not earn credit for citing this as a special case of a more general result.
%
%\item \label{Ch12_Xbar} Let $\bar{X}$ be the random variable for the sample mean, $\bar{X}=\frac{\sum_{i=1}^nX_i}{n}$, where the $X_i$ are i.i.d. random variables with common mean $\mu$ and variance $\sigma^2$.
%	\begin{enumerate}
%	\item Find $\mathbb{E}[\bar{X}]$.
%	\item Find $Var[\bar{X}]$.
%	\end{enumerate}
%
%\item \label{Ch12_p_hat} Let $\hat{p}$ be the random variable for the sample proportion, $\hat{p}=\frac{X}{n}$, where $X$ is the number of successes in a random sample of size $n$. Assume the probability of success is $p$.
%	\begin{enumerate}
%	\item Find $\mathbb{E}[\hat{p}]$.
%	\item Find $Var[\hat{p}]$.
%	\end{enumerate}



%-----------------------------------
% Chapter 2 or Chapter 15 (Binomial Probability)     
%-----------------------------------
%\item \label{ParkinsonsBinom} Read the Washington Post article \emph{The amazing woman who can smell Parkinson's disease - before symptoms appear}
%(\url{http://www.washingtonpost.com/news/morning-mix/wp/2015/10/23/scottish-woman-detects-a-musky-smell-that-could-radically-improve-how-parkinsons-disease-is-diagnosed/})
%
%\medskip
%
%Assuming Joy Milne does not have the ability to detect Parkinson's disease via smell, answer the following questions: 
%	\begin{enumerate}
%	\item What is the probability of her correctly detecting Parkinson's by smelling one t-shirt?
%	\item What is the probability of her correctly detecting Parkinson's in 12 out of 12 t-shirts?
%	\end{enumerate}

%-----------------------------------
% Chapter 15 (Binomial Probability)          \label{NTB_15}
%-----------------------------------

%\item \label{SumBinom} Let $X_i\sim$ Binomial($n_i,p$) be independent r.v.'s for $i=1,\ldots,m$. 
%	\begin{enumerate}
%	\item What does the r.v. $X=\sum_{i=1}^mX_i$ count, and what is the distribution of $X$? Make sure to specify the parameters of $X$'s distribution.
%	\item Find $\mathbb{E}[X]$. \emph{Make sure to show your work for (b) and (c). However, you may use without proof what you know about the mean and variance of each $X_i$.}
%	\item Find $Var[X]$.
%	\end{enumerate}


%-----------------------------------
% Chapter 17 (Negative Binomial Probability)          \label{NTB_17}
%-----------------------------------

%\newpage
%
%\item \label{SumNegBinom} Let $X_i\sim$ Negative Binomial($r_i,p$) be independent r.v.'s for $i=1,\ldots,m$. 
%	\begin{enumerate}
%	\item What does the r.v. $X=\sum_{i=1}^mX_i$ count, and what is the distribution of $X$? Make sure to specify the parameters of $X$'s distribution.
%	\item Find $\mathbb{E}[X]$. \emph{Make sure to show your work for (b) and (c). However, you may use without proof what you know about the mean and variance of each $X_i$.}
%	\item Find $Var[X]$.
%	\end{enumerate}


%-----------------------------------
% Calculus Review (pre Ch 24)          \label{NTB_Calc}
%-----------------------------------
%
%\item \label{CalculusReview} Calculus Review %These will not be collected.  
%
%
%\begin{enumerate}
%\item $$\int_0^yc(x+y)dx$$
%\item $$\frac{d}{dx}\bigg(\frac{4}{9}x^2y^2+\frac{5}{9}xy^4\bigg) $$
%\item $$\frac{d}{dy}\bigg(\frac{4}{9}x^2y^2+\frac{5}{9}xy^4\bigg) $$
%
%\item $$\int_0^y2e^{-x}e^{-y}dx$$
%\item $$\int_0^\infty xye^{-(x+y)}dy$$
%\item $$\int_x^{2x} 2e^{-(x+3y)}dy$$
%
%\item Find the area of the region bounded by the graphs of $f(x)=2-x^2$ and $g(x)=x$ by integrating with respect to $x$.
%\item Find the area of the region bounded by the graphs of $f(x)=2-x^2$ and $g(x)=x$ by integrating with respect to $y$.
%
%\item Find the area of the region bounded by the graphs of $x=3-y^2$ and $y=x-1$ by integrating with respect to $x$.
%\item Find the area of the region bounded by the graphs of $x=3-y^2$ and $y=x-1$ by integrating with respect to $y$.
%\end{enumerate}








%------------------------------------------------------------------
\end{enumerate}  % for all NTB
%------------------------------------------------------------------


%------------------------------------------------------------------
% ANSWERS          \label{ANSWERS}
%------------------------------------------------------------------

\rule{500pt}{1pt}
\bigskip

Selected answers (or hints) not provided at the end the book:
\begin{itemize}  % ANSWERS

% don't always assign #12
%\item Chapter 1          \label{ANS_1}
%	\begin{itemize}
%	\item \# 12  AITS
%	\end{itemize}

%\item Chapter 2          \label{ANS_2}
%	\begin{itemize}
%	\item  \# 4.  0.35
%	\item  \# 8. 0.03125
%	\item  \# 16 0.48
%	\item  \# 30 (a) 0.189  \ \ \ \ (b) 0.811 \ \ \ \ (c)  0.189
%	\end{itemize}
%
%
%\item Chapter 22          \label{ANS_22}
%	\begin{itemize}
%	\item  \# 30. (a) 2,835  \ \ \ \ (b) 405 \ \ \ \  (c) 10,780  \ \ \ \ (d) 7,980 
%	\item  \# 40. 0.6666667
%	\item  \# 42. 0.002116402  (This is the answer when $n=5$. Your answer needs to be in terms of $n$.)
%	\item  \# 44. 0.3
%	\item  \# 46. 0.3333333
%	\item  \# 48. 0.007936508  (This is the answer when $n=5$. Your answer needs to be in terms of $n$.)
%	\end{itemize}

%\item Chapter 3          \label{ANS_3}
%	\begin{itemize}
%	\item[\#4]  (a) 0.111328\ \ \ \ (b) 0.004872\ \ \ \ 0.995128
%	\item[\#10] (c)  0.384 If you have the right answer to (c), then you should be able to figure out the rest (see (e)).
%%	\item  \# 13 Round your answer to the next greatest integer. The answer is an integer between 5 and 10 inclusive.
%	\item[\#12] No.
%	\item[NTB \#\ref{Ch_3_pens}]  (a) 0.0799 \ \ \ \ (b) 0.07553 \ \ \ \ (c) 0.0655
%	\end{itemize}
%
%\item Chapter 4          \label{ANS_4}
%	\begin{itemize}
%	\item[\#4] 0.25
%	\item[\#12]   (a) 0.4285714  \ \ \ \ (b) 0.4285714 \ \ \ \ (c)  0.1428571
%	\end{itemize}
%
%\item Chapter 5           \label{ANS_5}
%	\begin{itemize}
%%	\item Non-textbook problem \ref{Basil}.
%%	\item  Non-textbook problem \ref{Enterprise}.
%%	\item[NTB \#\ref{Ch5_UrnMultiplicationRule}]   (a) 0.005\ \ \ \  (b) $2.78\times10^{-6}$
%%	\item[NTB \#\ref{Ch5_politics}]   0.43
%	\item[NTB \#\ref{Ch5_Bayes_F16Ex1}]   0.392
%	\item[NTB \#\ref{Ch5_Bayes_F05Ex1_solve_for_n}]  11.89 (rounds to about 12)	
%	\end{itemize}



%\item Chapter 7          \label{ANS_7}
%	\begin{itemize}
%	\item[\# 2] \ \   $X\in(0,\infty)$, continuous; $Y\in\{0,1,2,\ldots\}$, discrete
%	\item[\# 10]\ \  $X_j\in[0,\infty),j=1,\ldots,100$; $Y\in[0,\infty)$; both continuous
%	\item[\# 16]\ \  $Y$ could be 0
%	\item[\# 18]\ \  Yes, a r.v. can be both. Give an example!
%	\end{itemize}
%
%\item Chapter 8          \label{ANS_8}
%	\begin{itemize}
%	\item[\# 2]  (a) $p(x)=\binom{7}{x}(.5)^7$ for $x=0,1,2,\ldots,7$
%	\item[\# 9] (a) $c = \frac{1}{8}$
%	\item[\# 10]  \ \ 
%        	\begin{center}
%        	\begin{tabular}{|c|c|c|c|c|}
%        	\hline	
%        	$x$ & 2 & 4 & 6 & 8\\
%        	\hline
%        	$p(x)$ & 3/10 & 1/2 & 3/20 & 1/20\\
%        	\hline
%        	\end{tabular}
%        	\end{center}
%	\end{itemize}
%
%\item[] Chapter 9           \label{ANS_9}
%\begin{itemize}
%%	\item[]  \# 2 \ \ $p_{X,Y}(x,y)=.0003(.99)^{x-1}(.97)^{y-1}$, for what values of $x$ and $y$?
%%	\item[]  \# 8 \ \ You don't need to figure out this formula for $p(x|y)$, just check your answer is consistent with it. For $x,y=1,2,\ldots,6$:
%%$$
%%p(x|y) = \left\{
%%        \begin{array}{ll}
%%            0 & \quad x > y \\
%%            \frac{y}{2y-1} & \quad x =y \\
%%            \frac{1}{2y-1} & \quad x <y 
%%        \end{array}
%%    \right.
%%$$
%     \item[NTB \# \ref{Ch9_Smoking_vs_Education}]  \ \ Partial answers: 
%	\begin{itemize}
%%	\item[] (a) $p_{X,Y}(X=\text{high school}, Y=\text{no smoking at all}) = 0.100$ 
%%	\item[] (d) $F_{X,Y}(X=\text{high school}, Y=\text{no smoking at all}) = 0.723$ 
%	\item[(g)] $p_{X|Y}(X=\text{high school}| Y=\text{no smoking at all}) = 0.476$ 
%	\item[(h)]  $p_{Y|X}( Y=\text{no smoking at all}|X=\text{high school}) = 0.200$ 
%	\end{itemize}
%\end{itemize}


\item[] Chapter 10          \label{ANS_10}
%\begin{center}
%  \includegraphics[width=5in]{Chpt6EvenAnswers.pdf}
%\end{center}
\begin{itemize}
	\item[\# 6] \ \   750.5
	\item[\# 8] \ \   0.9
	\item[\# 10] \ \  201
	\item[ \# 14]\  \  (a)  1.875 \ \ \ \ (b) 3.125 \ \ \ \ 
\end{itemize}


\item[] Chapter 11          \label{ANS_11}
	\begin{itemize}
	\item[\# 2] \ \  1.6
%	\item[\# 12] \ \  $\approx$ 2.737
	\item[\# 18] \ \  a)  48.5 \ \ \ \ (b) 96 \ \ \ \ 
	\item[\# 20] \ \  $\approx$ 23.077	
\end{itemize}


%\item[] Chapter 12           \label{ANS_12}
%	\begin{itemize}
%	\item[\# 2]   \ \  64.8
%	\item[\# 12]   \ \ 1,096,357
%	\end{itemize}


%\item[] Chapter 13          \label{ANS_13}
%%\begin{center}
%%  \includegraphics[width=5in]{Chpt6EvenAnswers.pdf}
%%\end{center}
%	\begin{itemize}
%	\item[\# 4] \ \  (a) 260/9  \ \ \ \ (b) 2.833 \ \ \ \   (c)  $2.679\times 10^{-5}$ \ \ \ \ (d) Same idea as (c) Replace 10's with 100.  \ \ \ \ 
%%	\item[] \# 5\ \   (a)  $p_X(x)=\frac{\binom{3}{x}\binom{17}{5-x}}{\binom{20}{5}}$, for $x=0,1,2,3$ \ \ \ \ (c) 0.75 \ \ \ \   (d) 0.5033
%	\item[\# 6] \ \   (a)  $p_X(x)=\binom{4}{x}.3^x .7^{4-x}$, for $x=0,1,\ldots,4$ \ \ \ \ (d) 0.3483 \ \ \ \   (e) 0.9163  \ \ \ \   (f) 0.0233  \ \ \ \   (g) 1
%
%	\item[\# 8] \ \   (a) T  \ \ \ \ (b) F \ \ \ \   (c) F  \ \ \ \ (d) F \ \ \ \  (e)  T \ \ \ \ (f)  T\ \ \ \   (g)  T
%	\item[\# 10] \ \ (a)  T \ \ \ \ (b)  T\ \ \ \   (c)  F \ \ \ \ (d)  T\ \ \ \  (e)  T \ \ \ \ (f) F\ \ \ \   (g)  T \ \ \ \ (h) T  (nonnegative instead of positive) \ \ \ \  (i) F 
%	\end{itemize}


%
%\item[] Chapter 20           \label{ANS_20}
%	\begin{itemize}
%	\item[\# 2]   \ \  (a) 0.0001  \ \ \ \ (b) Discrete since $X$ has a finite number of possible values. Uniform since each outcome is equally likely. \ \ \ \   (c) $X$ = randomly selected 4-digit ID\#; $X=0000,0001,\ldots,9999$   \ \ \ \ (d) 5000.5 \ \ \ \  (e) 8,333,333.25
%	\end{itemize}
%
%\item[] Chapter 15           \label{ANS_15}
%	\begin{itemize}
%	\item[\# 18]   \ \  (a) Bin(21,0.65)  \ \ \ \ (b) 4.78 \ \ \ \ 
%	\end{itemize}
%
%\item[] Chapter 16           \label{ANS_16}
%	\begin{itemize}
%	\item[ \# 8]  \ \  (c)   $1.03\times 10^{-6}$\ \ \ \ (d)  10 questions: 91.43 minutes\ \ \ \ 
%	\end{itemize}
%
%\item[] Chapter 17           \label{ANS_17}
%	\begin{itemize}
%	\item[\# 6]   \ \  (a)  400, 87.18 \ \ \ \ (b) No \ \ \ \ 
%	\item[\# 12]   \ \  (c) 0.8000
%	\end{itemize}
%
%\item[] Chapter 19           \label{ANS_19}
%	\begin{itemize}
%	\item[\# 6]   \ \  (c) 15.625  \ \ \ \ (d) 0.0486 \ \ \ \  (f)  0.0488
%
%	\item[\# 18]   \ \  100
%	\end{itemize}
%
%\item[] Chapter 18           \label{ANS_18}
%	\begin{itemize}
%	\item[\# 24]   \ \  (c) 0.8571
%	\item[\# 26]   \ \  162,754.8
%	\end{itemize}
%
%
%\item[] Calculus Review           \label{ANS_CALC}
%	\begin{itemize}
%	\item[(a)] \ \  $c(\frac{y^{2}}{2}+y^{2})$
%	\item[(b)] \ \  $ \frac{8}{9}xy^{2}+\frac{5}{9}y^{4}  $
%	\item[(c)] \ \  $ \frac{8}{9}x^{2}y+\frac{20}{9}xy^{3}  $ 
%	\item[(d)] \ \   $ -2e^{-2y}+2e^{-y}  $
%	\item[(e)] \ \   $  xe^{-x} $
%	\item[(f)] \ \   $-\frac{2}{3}(e^{-7x}-e^{-4x})$
%	\item[(g)] \ \   $  \frac{9}{2} $
%	\item[(h)] \ \   $ \frac{9}{2}  $
%	\item[(i)]  \ \   $ \frac{9}{2}  $
%	\item[(j)] \ \   $  \frac{9}{2} $
%	\end{itemize}
%	
%	
%	
%\item Chapter 24           \label{ANS_24}
%	\begin{itemize}
%	\item  \# 2 (a)  Discrete \ \ \ \ (b) Discrete \ \ \ \   (c)  Continuous
%%	\item  \# 4 Filling in the table row-by-row: ?, 0.9, 0.1, 0.1, ?, 0.1, ?, 0
%	\item  \# 22 
%	$$
%f_X(x) = \left\{
%        \begin{array}{ll}
%            0 & \quad x <0 \\
%            \frac{7x}{4} & \quad 0\leq x\leq 1 \\
%            0 & \quad 1< x< 7 \\
%            \frac{1}{8} & \quad 7\leq x\leq 8 \\
%            0 & \quad  x>8 \\
%        \end{array}
%    \right.
%$$
%
%	\end{itemize}
%
%



\end{itemize}  % ANSWERS
\end{document}




	\begin{itemize}
	\item  \#
	\item  \#
	\item  \#
	\end{itemize}

(a)   \ \ \ \ (b)  \ \ \ \ 


