\documentclass[12pt]{article}
%%%%%%%%%%%%%%%%%%%%%%%%%%%%%%%%%%%%%%%%%%%%%%%%%%%%%%%%%%%%%%%%%%%%%%%%%%%%%%%%%%%%%%%%%%%%%%%%%%%%%%%%%%%%%%%%%%%%%%%%%%%%%%%%%%%%%%%%%%%%%%%%%%%%%%%%%%%%%%%%%%%%%%%%%%%%%%%%%%%%%%%%%%%%%%%%%%%%%%%%%%%%%%%%%%%%%%%%%%%%%%%%%%%%%%%%%%%%%%%%%%%%%%%%%%%%
\usepackage{amsmath}
\usepackage{amssymb}
\usepackage{hyperref}
\usepackage{graphicx}
\setcounter{MaxMatrixCols}{10}
%TCIDATA{TCIstyle=LaTeX article (bright).cst}

%TCIDATA{OutputFilter=LATEX.DLL}
%TCIDATA{Version=5.50.0.2890}
%TCIDATA{<META NAME="SaveForMode" CONTENT="1">}
%TCIDATA{BibliographyScheme=Manual}
%TCIDATA{Created=Thursday, November 10, 2005 15:20:28}
%TCIDATA{LastRevised=Monday, August 31, 2009 17:44:14}
%TCIDATA{<META NAME="GraphicsSave" CONTENT="32">}
%TCIDATA{<META NAME="DocumentShell" CONTENT="Exams and Syllabi\SW\Assignment">}

\setlength{\topmargin}{-1.0in}
\setlength{\textheight}{9.25in}
\setlength{\oddsidemargin}{0.0in}
\setlength{\evensidemargin}{0.0in}
\setlength{\textwidth}{6.5in}
\def\labelenumi{\arabic{enumi}.}
\def\theenumi{\arabic{enumi}}
\def\labelenumii{(\alph{enumii})}
\def\theenumii{\alph{enumii}}
\def\p@enumii{\theenumi.}
\def\labelenumiii{\arabic{enumiii}.}
\def\theenumiii{\arabic{enumiii}}
\def\p@enumiii{(\theenumi)(\theenumii)}
\def\labelenumiv{\arabic{enumiv}.}
\def\theenumiv{\arabic{enumiv}}
\def\p@enumiv{\p@enumiii.\theenumiii}
\pagestyle{plain}
\setcounter{secnumdepth}{0}
\parindent=0pt
%\input{tcilatex}
\begin{document}


\begin{center}

Homework \#7

BSTA 550

Due: Saturday, November 20, 2021 

%\bigskip
%
%\textit{Due to the Thanksgiving holiday,}  \newline 
%\textit{this assignment must be turned in by 5 pm on Wednesday.}

%\bigskip
%
\bigskip


% \label{HW1_odd_answers}\textbf{The answers to odd exercises are in the back of the book} 
%\newline \textbf{starting on pg. 621.}

\end{center}


\bigskip

Complete all of the problems listed below. \newline 
Only turn in the ones listed in the ``Turn In" column. \newline
%Please submit problems in the order they are listed and please staple your assignment.
Please submit problems in the order they are listed.

\bigskip

\textit{You must show all of your work to receive credit. Don't forget to define every r.v. you use!}  \newline 
\textit{In particular, \textbf{if a similar problem was done in class or an example in the book, make sure to still show every step in the solution and not just cite the examples' results}.}  \newline 



%-----------------------------------
% Assignment          \label{Assignment}
%-----------------------------------

\begin{center}
\begin{tabular}{|c|c||c|}
\hline
Chapter & Turn In & Extra Problems\\
\hline

%%Calculus Review &  NTB\# \ref{CalculusReview} (g)-(j)  & NTB\# \ref{CalculusReview} (a)-(f)\\  
%Calculus Review &    & NTB\# \ref{CalculusReview} \\  
%\hline

%%% Ch 24 461 F15 changed 4 to 3; added 18 to turn in
%%% Ch 24 550 F15 Added 20 to turn in with additional parts and moved 18 to list
%24 &  \# 19, 20*  & \# 2, 3, 7, 17, 18, 22, 23\\  
%\hline
%%R & NTB \# \ref{Rsim} below  & \\ % From HW 3 in F15
%%\hline

% Ch 25 F20 HW 7
%25 & \# 18  & \# $1, 4^{\dagger}, 8^{\dagger}, 17^{\dagger}, 23, 24$\\


% F21 Ch 25 Part 1
%25 & \# 18  & \# $1, 4, 8, 23, 24$\\

% F21 Ch 25 Part 2
% 17: max
25 &  & \# $17$\\

\hline	

%F19: added a few Ch 26 problems, although covering Ch 26 on Monday
% since CH 26 has good joint pmf examples
%26* & \# 12,  NTB \#\ref{pdfMax}  & \\
%\hline

26* & \# 12,  NTB \#\ref{pdfMax}, NTB \#\ref{Ch26_fxy_Zmin}  & \# $7, 9, 19, 20$\\
\hline	

25 & NTB \#\ref{Ch25_XYratio}  & \\
\hline

%26 & NTB \#\ref{Ch26_fxy_Zmin}  & \# 7, 9, 19, 20\\
%\hline	

27 &  \# 12**  & \# $6, 8, 13, 17$\\
\hline	
%
%28 & \# 18  & \# 1, 10\\
%\hline	
%
%29 ** & \# 26, NTB \ref{Ch29ExpRV}, \ref{Ch29UnifWaitingTime}   & \# 10, 14, 23, 11, 13, 32\\  % moved , NTB \ref{Ch29UnifWaitingTime} to Ch31
%%\ref{Ch29_SumsRV}
%\hline	
%
%30 &   & \# 4, 7-12\\
%\hline

%31 &  \# 18  & \# 13, 14, 17\\   % moved to Ch29: NTB \ref{Ch29UnifWaitingTime}
%\hline
%
%32 & \# 8 & \# 3, 5, 10*, 15\\
%\hline	
%
%33  &  NTB \ref{Ch33PoissonProcess}   & \# 3, 9, 10\\
%\hline	
%
%35 & \# 10, NTB \# \ref{Ch35_Normal_Parachute}    & \# 6, 9, 24\\   % 461 also assigned \# 35.10 to turn in
%\hline	
	
%Ch 43 split up into two assignments
%   part 1 of Ch43
%\# 1-4: Geo: mgf, mean & variance
%\# 9-12: Exp: mgf, mean & variance
%43 &  \# 9**, 10***, 11, 12***, NTB \# \ref{Ch43_R_Var}  & \# 1-4\\
%\hline	

%   part 2 of Ch43
%43 &  NTB \# \ref{Ch43_SumExpGamma} \&  \ref{Ch43_SumChiSquareOneDF}  & \\
%\hline	
%
%36 &  \# 12*, 14  & \# 4, 11, 13, 15, 16\\
%\hline	
%
%37 & \# 24, 30 & \# 2, 4, 13, 20, 29\\  %F15: collecting 24 (Poisson) instead of 4 (mean)
%\hline	


\end{tabular}
\end{center}


%-----------------------------------
% Comments \label{Comments}
%-----------------------------------
%-----------------------------------
% General: breakout sessions          \label{COM_breakouts}
%-----------------------------------

%${\dagger}$ indicates problems we will be going over in class
\vspace{.1cm}

%-----------------------------------
% Chapter 24          \label{COM_24}
%-----------------------------------

%* (Ch 24) Also find the cdf $F_X(x)$.

%\smallskip

%-----------------------------------
% Chapter 26          \label{COM_26}
%-----------------------------------

% F19
%* I added a few Chapter 26 problems since they primarily cover the material from Chapter 25. All you need to know from Chapter 26 to do these problems is that continuous r.v.'s $X$ and $Y$ are independent if and only if $f_{X,Y}(x,y) = f_{X}(x)f_{Y}(y)$. We will cover Chapter 26 on Monday. I will add the rest of Chapter 26 problems to the next homework assignment.

% F20
* Although within Chapter 26, these exercises are primarily practicing the material from Chapter 25. 



%-----------------------------------
% Chapter 27         \label{COM_27}
%-----------------------------------

%Ch 27 \# 12
\vspace{.1cm}
** For Ch 27 \# 12, in order to find the conditional densities in parts (a) and (b), you will need to calculate $f_Y(y)$ for the specific regions of $y$ specified. \newline
 After finding the conditional densities in parts (a) and (b), also calculate the conditional probabilities below. Please submit these together with your other work in parts (a) and (b): 
\begin{itemize}
\item[(a)]   Find $\mathbb{P}[0.5 < X < 3 | Y = 4]$.
\item[(b)]   Find $\mathbb{P}[0.5 < X < 3 | Y = 7]$.
\end{itemize}



%-----------------------------------
% Chapter 29    \label{COM_29}
%-----------------------------------

%\bigskip

%** I recommend doing the Extra Problems in the order listed.

%-----------------------------------
% Chapter 32    \label{COM_32}
%-----------------------------------

%* Assume $X$ and $Y$ are independent.

%-----------------------------------
% Chapter 43
%-----------------------------------

%\bigskip
%
%** Include in your answer an explanation as to why we need the condition that $t<\lambda$.
%
%\bigskip
%%\smallskip
%
%*** Do parts (a)-(c) below for \#10 and \#12:
%\begin{itemize} 
%\item[(a)] Answer the question using the mgf $M_X(t)$ as instructed in the book.
%\item[(b)] Answer the question using  $R_X(t)$ (as defined in class, and NTB \ref{Ch43_R_Var} below).
%\item[(c)] Which method did you prefer? Why?
%\end{itemize} 



%-----------------------------------
% Chapter 36    \label{COM_36}
%-----------------------------------

%* Assume the distances between the cars are independent.





%-----------------------------------
% End Comments
%-----------------------------------

\rule{500pt}{1pt}
%\vspace{.1cm}


 Non-textbook problems (NTB):          \label{NTB}

\begin{enumerate}  % for all NTB

%-----------------------------------
% Calculus Review (pre Ch 24)          \label{NTB_Calc}
%-----------------------------------

%\item[] \label{CalculusReview} Calculus Review %These will not be collected.  
%
%
%\begin{enumerate}
%\item $$\int_0^yc(x+y)dx$$
%\item $$\frac{d}{dx}\bigg(\frac{4}{9}x^2y^2+\frac{5}{9}xy^4\bigg) $$
%\item $$\frac{d}{dy}\bigg(\frac{4}{9}x^2y^2+\frac{5}{9}xy^4\bigg) $$
%
%\item $$\int_0^y2e^{-x}e^{-y}dx$$
%\item $$\int_0^\infty xye^{-(x+y)}dy$$
%\item $$\int_x^{2x} 2e^{-(x+3y)}dy$$
%
%\item Find the area of the region bounded by the graphs of $f(x)=2-x^2$ and $g(x)=x$ by integrating with respect to $x$.
%\item Find the area of the region bounded by the graphs of $f(x)=2-x^2$ and $g(x)=x$ by integrating with respect to $y$.
%
%\item Find the area of the region bounded by the graphs of $x=3-y^2$ and $y=x-1$ by integrating with respect to $x$.
%\item Find the area of the region bounded by the graphs of $x=3-y^2$ and $y=x-1$ by integrating with respect to $y$.
%\end{enumerate}
%


%-----------------------------------
% Chapter 25    \label{NTB_25}
%-----------------------------------

%F19: Put $Z=X/Y$ problem at end since most difficult

% F05 Exam 2 #4; Similar to L&M 3rd Ed #3.7.14
% Requires 2 cases for z
%\item \label{Ch25_XYratio}  Suppose that the random variables $X$ and $Y$ have joint density
%$f_{X,Y}(x,y)$, for $0<x<1$, and $\frac{1}{2}<y<1$. Set up the equation for
%the cdf of $Z$, where $Z=X/Y$.

%\emph{Hint: First determine what the possible values for $Z$ are. Then make a sketch of the domain of the joint pdf and shade in the region representing the cdf of Z for different values of $z$. Make sure to pay close attention to how the region we need to integrate over changes as $z$ changes. The cdf has two different cases depending on the value of $z$. Plug in specific values of $z$ and shade in the region representing the cdf to see why two different cases are needed.}




%-----------------------------------
% Chapter 26    \label{NTB_26}
%-----------------------------------

\item \label{pdfMax} Let $X_1, X_2, \ldots, X_n$ be i.i.d. random variables with common pdf $f_X(x)$ and cdf $F_X(x)$. Find the pdf for the random variable $Z$, where $Z = max(X_1, X_2, \ldots, X_n)$.


%%F12 Exam 2, last problem
\item \label{Ch26_fxy_Zmin}  Let $X$ and $Y$ be independent random variables with respective pdf's $f_X(x)=\frac{1}{5}$, for $0\leq x\leq 5$, and $f_Y(y)=2e^{-2y}$, for $ y>0$. 
\begin{enumerate}
\item Find the joint distribution $f_{X,Y}(x,y)$. %\vspace*{1in}
\item Find the probability that $X$ is less than $Y$.%\vspace*{2in}
\item Let $Z$ be the random variable that is the smaller of $X$ and $Y$. Find the cumulative distribution function for $Z$. 
\item Find the pdf for Z. %\newline 
\end{enumerate}


%-----------------------------------
% Chapter 25    \label{NTB_25_multi case}
%-----------------------------------
% F05 Exam 2 #4; Similar to L&M 3rd Ed #3.7.14
% Requires 2 cases for z
\item \label{Ch25_XYratio}  Suppose that the random variables $X$ and $Y$ have joint density
$f_{X,Y}(x,y)$, for $0<x<1$, and $\frac{1}{2}<y<1$. Set up the equation for
the cdf of $Z$, where $Z=X/Y$.

\emph{Hint: First determine what the possible values for $Z$ are. Then make a sketch of the domain of the joint pdf and shade in the region representing the cdf of Z for different values of $z$. Make sure to pay close attention to how the region we need to integrate over changes as $z$ changes. The cdf has two different cases depending on the value of $z$. Plug in specific values of $z$ and shade in the region representing the cdf to see why two different cases are needed.}



%-----------------------------------
% Chapter 28-29    \label{NTB_28-29}
%-----------------------------------

%%-----------------------------------
%% Also Chapter 31 (uniform)
%\begin{enumerate}
%\item \label{Ch29UniformRV} Let $f_X(x)=\frac{1}{b-a}$ for $a\leq x\leq b$. 
%	\begin{enumerate}
%	\item Show $E[X]=\frac{a+b}{2}$. 
%	\item Show $Var[X]=\frac{(b-a)^2}{12}$.
%	\end{enumerate}

%-----------------------------------
%% Also Chapter 32 (exponential)
%\item \label{Ch29ExpRV} Let $f_X(x)=\lambda e^{-\lambda x}$ for $x>0$, where $\lambda>0$. 
%	\begin{enumerate}
%%	\item Show $E[X]=\frac{1}{\lambda}$. 
%	\item Show $Var[X]=\frac{1}{\lambda^2}$. You may use the result from class for $\mathbb{E}[X]$ without first proving it.
%	\end{enumerate}

%% D&B 6.3 #27
%\item  \label{Ch29_SumsRV} A shipping company handles containers in three different sizes: (1) 27 $ft^3$ (3 x 3 x 3), (2) 125 $ft^3$, and (3) 512 $ft^3$. Let $X_i$ ($i =	1, 2, 3$) denote the number of type $i$ containers shipped during a given week. Suppose that $\mu_1 =200,\sigma_1=10,\mu_2 =250,\sigma_2=12,\mu_3 =100,\sigma_3=8$.
%
%	\begin{enumerate}
%	\item Assuming that $X_1,X_2,X_3$ are independent, calculate the expected value and variance of the total volume shipped.
%	\item Would your calculations necessarily be correct if the $X_i$'s were not independent? Explain.   
%	\end{enumerate}



%-----------------------------------
% Chapter 28-29 (& 31 since uniform, but already covered uniform r.v.'s
%-----------------------------------
%
%% D&B 6.3 #33
%\item \label{Ch29UnifWaitingTime} Suppose your waiting time for a bus in the morning is uniformly distributed on [0, 8], whereas waiting time in the evening is uniformly distributed on [0, 10] independent of morning waiting time. \newline
%\emph{Make sure to FIRST set up an equation for calculating the total waiting time in each question before calculating the mean and variance of the total waiting time.} \newline
%\emph{You may use results from class for the expected value and variance of uniform r.v.'s without proving them}
%	\begin{enumerate}
%	\item If you take the bus each morning and evening for a week, what is your total expected waiting time? 
%	\item What is the variance of your total waiting time?   
%	\item What are the expected value and variance of the difference between morning and evening waiting times on a given day? 
%	\item What are the expected value and variance of the difference between total morning waiting time and total evening waiting time for a particular week?
%	\end{enumerate}



%-----------------------------------
% Chapter 33   \label{NTB_33}
%-----------------------------------

%%F08 Exam 2
%\item  \label{Ch33PoissonProcess} Suppose that voters arrive at a polling station at the rate
%of 120 per hour.\newline For each of the following parts, \underline{give the
%name and parameter(s) of the distribution} to be used to model the event and \underline{set up
%the expression} to find the specified probability.\newline\emph{You do not
%need to compute the probability.}
%
%	\begin{enumerate}
%	\item The probability that the next voter will arrive in less than 30 seconds. 
%	\item The probability that 200 voters will arrive within two hours of each other.
%	\item The probability that the $50^{th}$ voter will arrive in between 15 and 30 minutes.
%	\end{enumerate}


%-----------------------------------
% Chapter 35     \label{NTB_35}
%-----------------------------------

%%DB 4.3 #50
%\item  \label{Ch35_Normal_Parachute}The automatic opening device of a military cargo parachute has been designed to open when the parachute is 200 m above the ground. Suppose opening altitude actually has a normal distribution with mean value 200 m and standard deviation 30 m. Equipment damage will occur if the parachute opens at an altitude of less than 100 m. What is the probability that there is equipment damage to the payload of at least one of the five independentIy dropped parachutes?  



%-----------------------------------
% Chapter 43     \label{NTB_43}
%-----------------------------------

%\item \label{Ch43_R_Var} Let $R_X(t)=\ln(M_X(t))$. Show that Var$(X)=R''_X(0)$.
%
%\item \label{Ch43_SumExpGamma} The mgf for a Gamma distribution is $M_X(t)=\frac{1}{(1-t/\lambda)^r}$. Use the mgf of an Exponential distribution (from \#43.9), to show that the sum of $n$ i.i.d.  Exponential($\lambda)$ random variables has a Gamma($r,\lambda$) distribution.
%
%
%\item \label{Ch43_SumPoisson} Use the mgf of a Poisson distribution to find the mgf of the following distributions. If the mgf is that of a common named distribution, then name the distribution and state its parameter(s).
%\begin{enumerate}
%\item The distribution of $\sum_{i=1}^nX_i$, if $X_i\sim$Poisson$(\lambda_i)$ and are independent.
%\item The distribution of $\sum_{i=1}^3X_i$, if $X_i\sim$Poisson$(\lambda)$ and are independent (i.i.d. in this case).
%\item The distribution of $3X$, if $X\sim$Poisson$(\lambda)$.
%\item Why are the answers to (b) and (c) different?
%
%\end{enumerate}
%
%\item \label{Ch43_SumChiSquareOneDF} Using mgf's, show that the sum of $n$ i.i.d. Chi Square random variables with one degree of freedom ($\chi^2_{(1)}$) r.v.'s has a Chi Square with $n$ degrees of freedom ($\chi^2_{(n)}$) distribution. 
%
%\emph{Hint:} First, look up the pdf of a $\chi^2_{(n)}$. This is a special case of the Gamma distribution with what parameters? Based on that and the information from \# \ref{Ch43_SumExpGamma} above, you can determine what the mgf of a $\chi^2_{(n)}$ is, which will help you determine whether the mgf of  the sum of $n$ i.i.d. $\chi^2_{(1)}$ r.v.'s has a $\chi^2_{(n)}$ distribution.
%



%------------------------------------------------------------------
\end{enumerate}  % for all NTB
%------------------------------------------------------------------


%------------------------------------------------------------------
% ANSWERS          \label{ANSWERS}
%------------------------------------------------------------------

\rule{500pt}{1pt}
\bigskip

%\newpage

Selected answers (or hints) not provided at the end the book:
\begin{itemize}  % ANSWERS



%\item[] Calculus Review           \label{ANS_CALC}
%	\begin{itemize}
%	\item[(a)] \ \  $c(\frac{y^{2}}{2}+y^{2})$
%	\item[(b)] \ \  $ \frac{8}{9}xy^{2}+\frac{5}{9}y^{4}  $
%	\item[(c)] \ \  $ \frac{8}{9}x^{2}y+\frac{20}{9}xy^{3}  $ 
%	\item[(d)] \ \   $ -2e^{-2y}+2e^{-y}  $
%	\item[(e)] \ \   $  xe^{-x} $
%	\item[(f)] \ \   $-\frac{2}{3}(e^{-7x}-e^{-4x})$
%	\item[(g)] \ \   $  \frac{9}{2} $
%	\item[(h)] \ \   $ \frac{9}{2}  $
%	\item[(i)]  \ \   $ \frac{9}{2}  $
%	\item[(j)] \ \   $  \frac{9}{2} $
%	\end{itemize}



%\item[Chapter 24]           \label{ANS_24}
%	\begin{itemize}
%	\item[2.]   (a)  Discrete \ \ \ \ (b) Discrete \ \ \ \   (c)  Continuous
%%	\item[4.]   Filling in the table row-by-row: ?, 0.9, 0.1, 0.1, ?, 0.1, ?, 0
%	\item[20.]  $k=660$   
%	\item[22.]   
%	$$
%f_X(x) = \left\{
%        \begin{array}{ll}
%            0 & \quad x <0 \\
%            \frac{7x}{4} & \quad 0\leq x\leq 1 \\
%            0 & \quad 1< x< 7 \\
%            \frac{1}{8} & \quad 7\leq x\leq 8 \\
%            0 & \quad  x>8 \\
%        \end{array}
%    \right.
%$$
%
%	\end{itemize}
%

%\item[Chapter 25]           \label{ANS_25}
%	\begin{itemize}
%	\item[4.]   \ \  7/16
%	\item[8.]   \ \  (a)  $\frac{25}{228} $ \ \ \ \ (b) $f_X(x)=\frac{1}{12}(x+1) $, for $0\leq x\leq 4$ \ \ \ \  
%	\newline  (c)  $f_Y(y)=\frac{3}{76}(y^2+1) $, for $0\leq y\leq 4$
%	\item[18.]   \ \  5/6
%	\item[24.]   \ \   (a)  $ f_X(x)=-2e^{-2x}+2e^{-x}  $, for $x\geq 0$ \ \ \ \  
%	  (b)  $ f_Y(y)=2e^{-2y}  $, for $ y\geq 0$
%	\end{itemize}

\item[Chapter 26]           \label{ANS_26}
	\begin{itemize}
	\item[12.]   %\ \  (a)  No. Why??? \ \ 
	\ \ (b) $ \frac{233}{256}  $ \ \ \ \ (c)  $ \frac{65}{256}  $ \ \ \ \ (d) $ \frac{1}{512}  $
	\item[20.]   \ \  (a)  Yes.  \ \ \ \ (b) $ \frac{15}{16}  $
	\item[NTB \ref{Ch26_fxy_Zmin}.]   \ \ (b) 0.09999546 \ \ (d) $f_Z(z) =\Big(\frac{11}{5} -  \frac{2z}{5}\Big)e^{-2z}$, for what values of $z$?
	\end{itemize}

\item[Chapter 27]           \label{ANS_27} 
	\begin{itemize}
	\item[6.]   $f_{X|Y}(x|y)=\frac{e^{-x/4-y/5}}{4(e^{-y/5}-e^{-9y/20})}$, for $0< x< y$
	\item[8.]   $f_{X|Y}(x|y)=\frac{1-x^2}{1-y-\frac{(1-y)^3}{3}}$, for $0\leq x, 0\leq y, x+y\leq 1$
	\item[12.]   (a) $f_{X|Y}(x|y)=\frac{1}{2}$ \ \ \ (c)  $\frac{4}{7}$
	%(a)  $f_{X|Y}(x|y)=\frac{1}{2}$, for $0\leq x\leq 2, 0\leq y \leq 6$ \newline
		%		\ \ \ \ (b) $f_{X|Y}(x|y)=\frac{3}{12-y}$, for $0\leq x\leq 4-y/3, 6\leq y \leq 12$ \ \ \ \   
	\end{itemize}


%\item[Chapter 28]         \label{ANS_28}
%	\begin{itemize}
%	\item[10.] (a) 8/9  \ \ \ \ (b) 14/3 \ \ \ \ 
%	\item[18.] 4/5
%	\end{itemize}

%\item[Chapter 29]         \label{ANS_29}
%	\begin{itemize}
%	\item[10.] (a)  26/81 \ \ \ \ (b)  74/9
%	\item[14.]	(a) 67/3  \ \ \ \ (b) 1/14 \ \ \ \   (c) 25/12  \ \ \ \ (d) $\sqrt{25/12}$
%	\item[26.]   250
%	\item[32.]  See notes (or book) for the proof from the discrete random variables case. The proof doesn't depend on what type of random variable (discrete vs. continuous) is being used.
%	\item[NTB \ref{Ch29UnifWaitingTime}.] (a) 63  \ \ \ \ (b) 287/3  \ \ \ \   (c) -1, 41/3  \ \ \ \ (d) -7, 287/3	
%	\end{itemize}

%\item[Chapter 30]         \label{ANS_30}
%	\begin{itemize}
%	\item[4.] $f_x(x)=1/2$ for $2\leq x\leq 4$
%	\item[8.] (a) T  \ \ \ \ (b) T \ \ \ \   (c)  F
%	\item[10.]	(a) F  \ \ \ \ (b)  T   
%	\item[12.]  (a) T  \ \ \ \ (b) T \ \ \ \   (c) F  \ \ \ \ (d) T
%	\end{itemize}

%\item[Chapter 31]         \label{ANS_31}
%	\begin{itemize}
%	\item[14.]  (a) 0.25  \ \ \ \ (b) 0.02887 \ \ \ \   (c)  0.063 \ \ \ \ (d) 0.0145 \ \ \ \  (e) 0.01625  \ \ \ \ (f)  0.0055    \ \ \ \ (f)  6.195\ \ \ \   (g) 0.00433  \ \ \ \ (h) 61.95 \ \ \ \   (i) 0.0433
%	\item[17.] 2.25
%	\item[18.] 7/15
%%	\item[NTB \ref{Ch29UnifWaitingTime}.] (a) 63  \ \ \ \ (b) 287/3  \ \ \ \   (c) -1, 41/3  \ \ \ \ (d) -7, 287/3
%	\end{itemize}
%
%\item[Chapter 32]         \label{ANS_32}
%	\begin{itemize}
%	\item[8.] 0.2526
%	\item[5.] 0.8047
%	\item[10.] 0.4323
%	\end{itemize}
%
%\item[Chapter 33]         \label{ANS_33}
%	\begin{itemize}
%	\item[10.] (a)  $f_x(x)=\frac{x}{9}e^{-x/3}$ for $ x> 0$ \ \ \ \ (b)  0.4963
%	\end{itemize}
%
%\item[Chapter 35]        \label{ANS_35}
%	\begin{itemize}
%	\item[6.] (a) 0  \ \ \ \ (b) -1.13  \ \ \ \   (c) $\pm 0.32$ 
%	\item[10.] (a) 0.0475  \ \ \ \ (b) 0.0475  \ \ \ \   (c)  0.2283 \ \ \ \ (d) 68.97 to 81.03 \ \ \ \  (e) 48 to 102  \ \ \ \  \newline (f)  68.97
%	\item[24.] (a) 0.2119  \ \ \ \ (b)  0.0011
%	\item[NTB \ref{Ch35_Normal_Parachute}.]\ \   0.002
%	\end{itemize}

%\item[Chapter 43]        \label{ANS_43}
%	\begin{itemize}
%	\item[]
%	%\item[1.] $M_X(t)=\frac{pe^t}{1-(1-p)e^t}$  % book corrected the answer
%	\item[NTB \ref{Ch43_SumPoisson}.]   (a)  Poisson$(\sum_{i=1}^n \lambda_i)$ \ \ \ \ (b) Poisson$(3\lambda)$ \ \ \ \   \newline
%	(c) $M_{3X}(t)=e{\lambda(e^{3t}-1)}$ This is not an mgf of a common probability distribution. 
%	 \ \ \ \ \newline 
%	 (d) In (b) we are adding independent r.v.'s $X_i$, while in (c) we are adding dependent r.v.'s ($3X=X+X+X$; $X$ is dependent with itself).
%	\end{itemize}




%\item[Chapter 36]        \label{ANS_36}
%	\begin{itemize}
%	\item[4.] 0.0044  
%	\item[12.]  (a) 0.9525  \ \ \ \ (b) 0.7939 \ \ \ \   (c) 0.7939  
%	\item[14.] 0.5911
%	\item[16.] (a)  $R=8.225\sigma+25\mu$ \ \ \ \ (b) $R=16.45\sigma+100\mu$ \ \ \ \   (c) $R=164.5\sigma+10,000\mu$  \ \ \ \ (d) $R=1.645\sqrt{n}\sigma+n\mu$
%	\end{itemize}
%
%\item[Chapter 37]        \label{ANS_37}
%	\begin{itemize}
%	\item[2.] 0.8869
%	\item[4.]  0.0023
%	\item[20.]  0.3936
%	\item[24.] 0.4562
%	\item[30.]  (b)  0.0022 \ \ \ \ (c)  $478.696\approx 479$
%	\end{itemize}





\end{itemize}  % ANSWERS
\end{document}




	\begin{itemize}
	\item[]  \#
	\item[]  \#
	\item[]  \#
	\end{itemize}

(a)   \ \ \ \ (b)  \ \ \ \ 


