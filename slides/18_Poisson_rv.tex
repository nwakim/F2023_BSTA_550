%\documentclass[12pt,letterpaper]{article}
\documentclass[12pt]{amsart}

\usepackage[left=1in, right=1in, top=.5in, bottom=.5in]{geometry}
\usepackage{latexsym,amssymb,amsmath,amsthm,amsopn,verbatim, mathpazo, graphicx, lmodern}
%\usepackage{latexsym,amssymb,amsmath,amsthm,amsopn,verbatim, mathpazo, graphicx}
\newcommand{\vs}{\vskip.5cm}
\newcommand{\hs}{\hskip1cm}
\newcommand{\ds}{\displaystyle}
\setlength{\parindent}{0pt}

\usepackage{hyperref}  % links, urls
\urlstyle{same}

\usepackage[T1]{fontenc}
\usepackage{libertine}
\renewcommand*\familydefault{\sfdefault}  %% Only if the base font of the document is to be sans serif


\newtheorem{theorem}{Theorem}[section]
\newtheorem{corollary}{Corollary}[theorem]
\newtheorem{lemma}[theorem]{Lemma}
\newtheorem{definition}[theorem]{Definition}
\newtheorem{example}[theorem]{Example}

\newcommand\indep{\protect\mathpalette{\protect\independenT}{\perp}}
\def\independenT#1#2{\mathrel{\rlap{$#1#2$}\mkern2mu{#1#2}}}

\usepackage{fancyhdr}
\pagestyle{fancy}
\fancypagestyle{plain}{}
%\fancyhead{} % clear all header fields
%\fancyhf{}
\rhead{BSTA 550 Ch 18}   % from fancyhdr package
\renewcommand{\headrulewidth}{1pt}
\renewcommand{\footrulewidth}{0pt}



\begin{document}


%----------------------------------------------------------------------------
\setcounter{section}{18}
{\huge  
\section*{Chapter 18: Poisson r.v.'s}
}
%----------------------------------------------------------------------------

%----------------------------------------------------------------------------
{\large 
%----------------------------------------------------------------------------

%----------------------------------------------------------------------------
%\vspace{18cm}
%\hrule
%\vspace{1cm}



\vspace{.5cm}

\textbf{Scenario:} We are counting the number of successes in a fixed time period, which has a constant rate of successes.

%----------------------------------------------------------------------------
\vspace{.5cm}
\hrule
\vspace{.5cm}

\begin{itemize}
\item Recall that if $X\sim Binomial(n,p)$, then 
	\begin{itemize}
	\item $X$ models the number of successes 
	\item in $n$ independent (Bernoulli) trials 
	\item that each have the same probability of success $p$.
	\end{itemize}
\item Poisson r.v.'s are similar, 
	\begin{itemize}
	\item except that instead of having $n$ discrete independent trials, 
	\item there is a fixed time period during which the successes happen.
	\end{itemize}
\end{itemize}

%----------------------------------------------------------------------------
\vspace{.5cm}
\hrule
\vspace{.5cm}

%----------------------------------------------------------------------------
\begin{example}\ %\newline
Some examples of Poisson r.v.'s:
\begin{itemize}
\item Number of visitors to an emergency room in an hour during a weekend night
\item Number of study participants enrolled in a study per week
\item Number of gun shootings in a square mile
\end{itemize}

\end{example}
%----------------------------------------------------------------------------
\vspace{.5cm}
\hrule
\vspace{.5cm}


\textbf{Properties of Poisson r.v.'s}


%****************************************************************************************
\newpage


%----------------------------------------------------------------------------
\begin{example}\ %\newline
Suppose an emergency room has an average of 50 visitors per day. Find the following probabilities.

\begin{enumerate}
\item Probability of 30 visitors in a day.

\textbf{Solution:}
\vspace{4cm}

\item Probability of 8 visitors in an hour.

\textbf{Solution:}
\vspace{5cm}

\item Probability of at least 8 visitors in an hour.

\textbf{Solution:}
\vspace{4.5cm}

\end{enumerate}


\end{example}
%----------------------------------------------------------------------------



%----------------------------------------------------------------------------
\begin{example}\ %\newline
Suppose emergency room 1 has an average of 50 visitors per day, and emergency room 2 has an average of 70 visitors per day, independently of each other. What is the probability distribution to model of the total number of visitors to both?
\end{example}
\textbf{Solution:}
%\vspace{4cm}


%\vspace{4cm}
%\hrule
%\vspace{.5cm}


%****************************************************************************************
\newpage

\begin{theorem}\ %\newline
If $X\sim Poiss(\lambda_1)$ and $Y\sim Poiss(\lambda_2)$ are independent of each other, then $Z=X+Y\sim Poiss(\lambda_1 + \lambda_2)$.
\end{theorem}

%----------------------------------------------------------------------------
\vspace{.5cm}
\hrule
\vspace{.5cm}


\textbf{Poisson vs. Binomial r.v.'s}
%----------------------------------------------------------------------------
\vspace{.5cm}

Both Poisson and Binomial r.v.'s are counting the number of successes

\begin{itemize}
\item If for a Binomial r.v. 
	\begin{itemize}
	\item the number of trials $n$ is very large, and 
	\item the probability of success $p$ is close to 0 or 1,
	\end{itemize}
\item then the Poisson distribution can be used to approximate Binomial probabilities.
\end{itemize}

%----------------------------------------------------------------------------
%\vspace{3cm}
%\hrule
%\vspace{.5cm}


%****************************************************************************************
\newpage
%----------------------------------------------------------------------------
\begin{example}\ %\newline
Suppose that in the long run, errors in a medical testing lab are made 0.1\% of the time. Find the probability that fewer than 4 mistakes are made in the next 2,000 tests.

\begin{enumerate}
\item Find the probability using the Binomial distribution.
\item Approximate the probability in part (1) using the Poisson distribution.
\end{enumerate}
\textbf{Solution:}

\end{example}
%----------------------------------------------------------------------------
%\vspace{4cm}
%\hrule
%\vspace{.5cm}



%----------------------------------------------------------------------------
}  % end large font
%----------------------------------------------------------------------------



\end{document}

