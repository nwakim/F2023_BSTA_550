%\documentclass[12pt,letterpaper]{article}
\documentclass[12pt]{amsart}

\usepackage[left=1in, right=1in, top=.5in, bottom=.5in]{geometry}
\usepackage{latexsym,amssymb,amsmath,amsthm,amsopn,verbatim, mathpazo, graphicx, lmodern}
%\usepackage{latexsym,amssymb,amsmath,amsthm,amsopn,verbatim, mathpazo, graphicx}
\newcommand{\vs}{\vskip.5cm}
\newcommand{\hs}{\hskip1cm}
\newcommand{\ds}{\displaystyle}
\setlength{\parindent}{0pt}

\usepackage{hyperref}  % links, urls
\urlstyle{same}

\usepackage[T1]{fontenc}
\usepackage{libertine}
\renewcommand*\familydefault{\sfdefault}  %% Only if the base font of the document is to be sans serif


\newtheorem{theorem}{Theorem}[section]
\newtheorem{corollary}{Corollary}[theorem]
\newtheorem{lemma}[theorem]{Lemma}
\newtheorem{definition}[theorem]{Definition}
\newtheorem{example}[theorem]{Example}

\newcommand\indep{\protect\mathpalette{\protect\independenT}{\perp}}
\def\independenT#1#2{\mathrel{\rlap{$#1#2$}\mkern2mu{#1#2}}}

\newcommand\Pbb{\mathbb{P}}
\newcommand\Ebb{\mathbb{E}}
\newcommand\pdfX{f_X(x)}
\newcommand\pdfY{f_Y(y)}
\newcommand\cdfX{F_X(x)}
\newcommand\cdfY{F_Y(y)}

\newcommand\intd{\displaystyle\int}


\usepackage{fancyhdr}
\pagestyle{fancy}
\fancypagestyle{plain}{}
%\fancyhead{} % clear all header fields
%\fancyhf{}
\rhead{BSTA 550 Calculus Review}   % from fancyhdr package
\renewcommand{\headrulewidth}{1pt}
\renewcommand{\footrulewidth}{0pt}



\begin{document}


%----------------------------------------------------------------------------
%\setcounter{section}{24}
{\huge  
\section*{Calculus Review }
}
%----------------------------------------------------------------------------

%----------------------------------------------------------------------------
{\large 
%----------------------------------------------------------------------------

%----------------------------------------------------------------------------
%\vspace{18cm}
%\hrule
%\vspace{1cm}



\vspace{.5cm}



\subsection{Differentiation}

%----------------------------------------------------------------------------
\begin{example} Find the derivatives of the following functions.

\begin{enumerate}

\item $f(x) = 2$

%	\textbf{Solution:}
\vspace{3cm}

\item $f(x) = 2x$

%\textbf{Solution:}
\vspace{3cm}

\item $f(x) = 2x+2$

%\textbf{Solution:}
\vspace{3cm}

\item $f(x) = x^2$

%\textbf{Solution:}
\vspace{3cm}


\item $f(x) = 3\sqrt{x}+\frac2x+5$

%\textbf{Solution:}
\vspace{3cm}

\item $f(x) = e^x$

%\textbf{Solution:}
\vspace{3cm}

\item $f(x) = \ln(x)$

%\textbf{Solution:}
\vspace{3cm}



\item $f(x) = x^2 e^x$

%\textbf{Solution:}
\vspace{5cm}


\item $f(x) = \frac{x^5}{2x+7}$

%\textbf{Solution:}
\vspace{5cm}



\item $f(x) = e^{-2x+7}$

%\textbf{Solution:}
\vspace{5cm}


\item $f(x) = \ln(x^2)$

%\textbf{Solution:}
\vspace{5cm}

\end{enumerate}

\end{example}
%----------------------------------------------------------------------------


%****************************************************************************************
\newpage
\subsection{Integration}
%\vspace{.1cm}
\subsubsection{Antidifferentiation}
%\vspace{.1cm}
%----------------------------------------------------------------------------
\begin{example} Find the antiderivatives of the following functions.
\begin{enumerate}
\item $f(x) = 2$
%	\textbf{Solution:}
%\vspace{2cm}
\item $f(x) = x$
%\textbf{Solution:}
\vspace{2cm}
\item $f(x) = \frac1x$

%\textbf{Solution:}
\vspace{3cm}

\item $f(x) = x^{3/2}$

%\textbf{Solution:}
\vspace{3cm}


\item $f(x) = e^x$

%\textbf{Solution:}
\vspace{3cm}

\item $f(x) = e^{-x}$

%\textbf{Solution:}
\vspace{3cm}

\item $f(x) = e^{-2x}$

%\textbf{Solution:}
\vspace{3cm}


\end{enumerate}

\end{example}
%----------------------------------------------------------------------------


%****************************************************************************************
\newpage
\subsubsection{Definite Integrals}
%\vspace{.1cm}
%----------------------------------------------------------------------------
\begin{example} Solve the following integrals.
\begin{enumerate}
\item $\intd_0^1 (2x+x^5)dx$
%	\textbf{Solution:}
\vspace{3cm}

\item $\intd_2^3 e^{-x}dx$
%	\textbf{Solution:}
\vspace{5cm}

%\item $\intd_0^x \lambda e^{-\lambda t}dt$
%%	\textbf{Solution:}
%\vspace{5cm}


\item $\intd_2^3 x e^{x^2}dx$
%	\textbf{Solution:}
\vspace{5cm}




\item $\intd_0^{\infty} x e^{-x}dx$
%	\textbf{Solution:}
\vspace{6cm}


%****************************************************************************************
\newpage
\item $\intd_1^2 x^2 \ln(x)dx$
%	\textbf{Solution:}
\vspace{7cm}

\item $\intd_1^2 \ln(x)dx$
%	\textbf{Solution:}
\vspace{7cm}

\item $\intd_1^2 x^2 e^{x}dx$
%	\textbf{Solution:}
\vspace{6cm}
\end{enumerate}

\end{example}
%----------------------------------------------------------------------------





%----------------------------------------------------------------------------
}  % end large font
%----------------------------------------------------------------------------



\end{document}

