%\documentclass[12pt,letterpaper]{article}
\documentclass[12pt]{amsart}

\usepackage[left=1in, right=1in, top=.5in, bottom=.5in]{geometry}
\usepackage{latexsym,amssymb,amsmath,amsthm,amsopn,verbatim, mathpazo, graphicx, lmodern}
%\usepackage{latexsym,amssymb,amsmath,amsthm,amsopn,verbatim, mathpazo, graphicx}



\newcommand{\vs}{\vskip.5cm}
\newcommand{\hs}{\hskip1cm}
\newcommand{\ds}{\displaystyle}
\setlength{\parindent}{0pt}

\usepackage{hyperref}  % links, urls
\urlstyle{same}

\usepackage[T1]{fontenc}
\usepackage{libertine}
\renewcommand*\familydefault{\sfdefault}  %% Only if the base font of the document is to be sans serif


\newtheorem{theorem}{Theorem}[section]
\newtheorem{corollary}{Corollary}[theorem]
\newtheorem{lemma}[theorem]{Lemma}
\newtheorem{definition}[theorem]{Definition}
\newtheorem{example}[theorem]{Example}

\newcommand\indep{\protect\mathpalette{\protect\independenT}{\perp}}
\def\independenT#1#2{\mathrel{\rlap{$#1#2$}\mkern2mu{#1#2}}}

\newcommand\Pbb{\mathbb{P}}
\newcommand\Ebb{\mathbb{E}}
\newcommand\gs{\sigma}
\newcommand\gl{\lambda}
\newcommand\ga{\alpha}
\newcommand\gb{\beta}
\newcommand\pmfX{p_X(x)}
\newcommand\pmfY{p_Y(y)}
\newcommand\pdfX{f_X(x)}
\newcommand\pdfY{f_Y(y)}
\newcommand\pdfXY{f_{X,Y}(x,y)}
\newcommand\cdfX{F_X(x)}
\newcommand\cdfY{F_Y(y)}
\newcommand\cdfXY{F_{X,Y}(x,y)}
\newcommand\pdfXgY{f_{X|Y}(x|y)}
\newcommand\pdfYgX{f_{Y|X}(y|x)}
\newcommand\mgfX{M_X(t)}
\newcommand\mgfY{M_Y(t)}


\newcommand\intd{\displaystyle\int}
\newcommand\intinft{\int_{-\infty}^{\infty}}

\usepackage{fancyhdr}
\pagestyle{fancy}
\fancypagestyle{plain}{}
%\fancyhead{} % clear all header fields
%\fancyhf{}
\rhead{BSTA 550 Ch 43}   % from fancyhdr package
\renewcommand{\headrulewidth}{1pt}
\renewcommand{\footrulewidth}{0pt}



\begin{document}


%----------------------------------------------------------------------------
\setcounter{section}{43}
{\huge  
\section*{Chapter 43: Moment Generating Functions \newline
Part 1}
}
%----------------------------------------------------------------------------

%----------------------------------------------------------------------------
{\large % begin large font
%----------------------------------------------------------------------------

%----------------------------------------------------------------------------
%\vspace{18cm}
%\hrule
%\vspace{1cm}



\vspace{.5cm}
%----------------------------------------------------------------------------
\begin{center}
\textbf{What are \textit{moments}?} 
\end{center}

\vspace{.5cm}
%----------------------------------------------------------------------------
\begin{definition} \ \  The $j^{th}$ moment of a r.v. $X$ is $\Ebb[X^j]$.
\end{definition}

\vspace{.5cm}
%----------------------------------------------------------------------------
\begin{example} \ \  $1^{st}-4^{th}$ moments.
\end{example}

\vspace{6cm}
\hrule
\vspace{.5cm}

%----------------------------------------------------------------------------
\begin{center}
\textbf{What is a \textit{moment generating function} (mgf)?} 
\end{center}

\vspace{.5cm}
%----------------------------------------------------------------------------
\begin{definition} \ \  If $X$ is a r.v., then 
$$
\mgfX = \Ebb[e^{tX}]
$$
is the \textbf{moment generating function} (\textbf{mgf}) associated with $X$.
\end{definition}

\vspace{1cm}
\textbf{Remarks} 

\begin{itemize}
\item For a discrete r.v., the mgf of $X$ is
$$
\mgfX = \Ebb[e^{tX}]=\sum_{all \ x}e^{tx}\pmfX
$$
\vspace{.5cm}
\item For a continuous r.v., the mgf of $X$ is
$$
\mgfX = \Ebb[e^{tX}]=\intinft e^{tx}\pdfX dx
$$

\vspace{.5cm}
\item The mgf $\mgfX$ is a function of $t$, not of $X$, and it might not be defined (i.e. finite) for all values of $t$. We just need it to be defined for $t=0$.
\end{itemize}


%****************************************************************************************
\newpage

\begin{example} \ \  What is $\mgfX$ for $t=0$?
\end{example}


\vspace{5cm}
\hrule
\vspace{.5cm}

%----------------------------------------------------------------------------
\begin{theorem} \ \  The moment generating function uniquely specifies a probability distribution.
\end{theorem}

\vspace{.5cm}
%----------------------------------------------------------------------------

\begin{theorem} \ \  
$$
\Ebb[X^r] = M_X^{(r)}(0)
$$
\end{theorem}

\begin{proof}
\end{proof}



%****************************************************************************************
\newpage

\begin{example} \ \  Let $X \sim Poisson(\gl)$. 

\begin{enumerate}
\item Find the mgf of $X$.

\vspace{8cm}

\item Find $\Ebb[X]$.

\vspace{6cm}

\item Find $Var(X)$.


\end{enumerate}

\end{example}


%****************************************************************************************
\newpage

\textbf{Remark} 

Finding the mean and variance is sometimes easier with the following trick.

\vspace{.5cm}
\begin{theorem}
Let $$R_X(t) = \ln[\mgfX]$$

Then,

$$\mu = \Ebb[X] = R_X'(0)$$
and
$$\gs^2 = Var(X) = R_X''(0)$$
\end{theorem}
\begin{proof}
\end{proof}


\vspace{6cm}
\hrule
\vspace{.5cm}


%----------------------------------------------------------------------------
\begin{example} \ \  Let $X \sim Poisson(\gl)$. 

\begin{enumerate}
\item Find $\Ebb[X]$ using $R_X(t)$.

\vspace{6cm}

\item Find $Var(X)$ using $R_X(t)$.


\end{enumerate}

\end{example}




%****************************************************************************************
\newpage

\begin{example} \ \  Let $Z$ be a standard normal random variable, i.e. $Z \sim N(0,1)$. 

\begin{enumerate}
\item Find the mgf of $Z$.

\vfill

\item Find $\Ebb[Z]$.

\vspace{4cm}

\item Find $Var(Z)$.

\vspace{3cm}
\end{enumerate}

\end{example}





%----------------------------------------------------------------------------
}  % end large font
%----------------------------------------------------------------------------



\end{document}

