%\documentclass[12pt,letterpaper]{article}
\documentclass[12pt]{amsart}

\usepackage[left=1in, right=1in, top=.5in, bottom=.5in]{geometry}
\usepackage{latexsym,amssymb,amsmath,amsthm,amsopn,verbatim, mathpazo, graphicx, lmodern}
%\usepackage{latexsym,amssymb,amsmath,amsthm,amsopn,verbatim, mathpazo, graphicx}
\newcommand{\vs}{\vskip.5cm}
\newcommand{\hs}{\hskip1cm}
\newcommand{\ds}{\displaystyle}
\setlength{\parindent}{0pt}

\usepackage{hyperref}  % links, urls
\urlstyle{same}

\usepackage[T1]{fontenc}
\usepackage{libertine}
\renewcommand*\familydefault{\sfdefault}  %% Only if the base font of the document is to be sans serif


\newtheorem{theorem}{Theorem}[section]
\newtheorem{corollary}{Corollary}[theorem]
\newtheorem{lemma}[theorem]{Lemma}
\newtheorem{definition}[theorem]{Definition}
\newtheorem{example}[theorem]{Example}

\newcommand\indep{\protect\mathpalette{\protect\independenT}{\perp}}
\def\independenT#1#2{\mathrel{\rlap{$#1#2$}\mkern2mu{#1#2}}}

\newcommand\Pbb{\mathbb{P}}
\newcommand\Ebb{\mathbb{E}}
\newcommand\pdfX{f_X(x)}
\newcommand\pdfY{f_Y(y)}
\newcommand\pdfXY{f_{X,Y}(x,y)}
\newcommand\cdfX{F_X(x)}
\newcommand\cdfY{F_Y(y)}
\newcommand\cdfXY{F_{X,Y}(x,y)}

\newcommand\intd{\displaystyle\int}

\usepackage{fancyhdr}
\pagestyle{fancy}
\fancypagestyle{plain}{}
%\fancyhead{} % clear all header fields
%\fancyhf{}
\rhead{BSTA 550 Ch 25}   % from fancyhdr package
\renewcommand{\headrulewidth}{1pt}
\renewcommand{\footrulewidth}{0pt}



\begin{document}


%----------------------------------------------------------------------------
\setcounter{section}{25}
{\huge  
\section*{Chapter 25: Joint densities}
}
%----------------------------------------------------------------------------

%----------------------------------------------------------------------------
{\large % begin large font
%----------------------------------------------------------------------------

%----------------------------------------------------------------------------
%\vspace{18cm}
%\hrule
%\vspace{1cm}



\vspace{.5cm}

Recall from Chapter 24, that the probability distribution, or \textbf{probability density function (pdf)}, of a continuous random variable $X$ is a function $f_X(x)$, such that for all real values $a,b$ with $a \leq b$,
$$
\mathbb{P}(a \leq X \leq b) = \int_a^b f_X(x)dx.
$$

%----------------------------------------------------------------------------


%----------------------------------------------------------------------------
\vspace{.5cm}
\hrule
\vspace{.5cm}


\textbf{How to define the joint pdf for continuous r.v.'s?}

%----------------------------------------------------------------------------
\vspace{9cm}
%\hrule
%\vspace{.5cm}

%----------------------------------------------------------------------------
\vspace{.5cm}
\hrule
\vspace{.5cm}

%----------------------------------------------------------------------------
\textbf{Remarks:}
\begin{enumerate}
\item Note that $\pdfXY \neq \mathbb{P}(X=x, Y=y)$!!!
\vspace{.5cm}
\item In order for $\pdfXY$ to be a pdf, it needs to satisfy the properties 
	\begin{itemize}
	\item $\pdfXY \geq 0$ for all $x,y$
	\item $\intd_{-\infty}^{\infty}\intd_{-\infty}^{\infty} \pdfXY dxdy=1$
	\end{itemize}
\end{enumerate}


%****************************************************************************************
\newpage



\textbf{Double Integrals Mini Lesson}

%----------------------------------------------------------------------------
\begin{example}\ Solve the following integrals.
\begin{enumerate}

\item $\intd_{2}^{3}\intd_{0}^{1} xy dydx$

%	\textbf{Solution:}
%\vspace{5cm}
\vfill

\item $\intd_{2}^{3}\intd_{0}^{1} (x+y) dydx$

%\textbf{Solution:}
\vfill

\item $\intd_{2}^{3}\intd_{0}^{1} e^{x+y} dydx$
\vfill

\end{enumerate}

\end{example}
%----------------------------------------------------------------------------


%****************************************************************************************
\newpage

%\textbf{Joint cdf's}
%----------------------------------------------------------------------------
\begin{definition}[Joint cumulative distribution function]\ \newline
The \textbf{joint cumulative distribution function (cdf)} of continuous random variables $X$ and $Y$, is the function $\cdfXY$, such that for all real values of $x$ and $y$,
$$
\cdfXY = \mathbb{P}(X \leq x, Y \leq y) = \int_{-\infty}^x\int_{-\infty}^y f_{X,Y}(s,t)dtds
$$
\end{definition}

%----------------------------------------------------------------------------
\textbf{Remarks:} \newline

\begin{itemize}
\item The definition above for $\cdfXY$ is a \textbf{function} of $x$ and $y$.
\item  The joint cdf at the point $(a,b)$, is 
$$
F_{X,Y}(a,b) = \mathbb{P}(X \leq a, Y \leq b) = \int_{-\infty}^a\int_{-\infty}^b f_{X,Y}(s,t)dtds
$$
\end{itemize}

%----------------------------------------------------------------------------
\vspace{.5cm}
\hrule
\vspace{.5cm}
%----------------------------------------------------------------------------

%----------------------------------------------------------------------------
%\textbf{Marginal pdf's} \newline

%----------------------------------------------------------------------------
\begin{definition}[Marginal pdf's]\ \newline
Suppose $X$ and $Y$ are continuous r.v.'s, with joint pdf $\pdfXY$. Then the \textbf{marginal probability density functions} are
\begin{eqnarray*}
\pdfX&=& \int_{-\infty}^{\infty} \pdfXY dy\\
\pdfY&=& \int_{-\infty}^{\infty} \pdfXY dx
\end{eqnarray*}

\end{definition}


%****************************************************************************************
\newpage

\begin{example}\ %\newline
Let $\pdfXY = \frac32 y^2$, for $0 \leq x \leq 2, \ 0 \leq y \leq 1$.

\vspace{.5cm}

\begin{enumerate}
\item Find $\Pbb(0 \leq X \leq 1, 0 \leq Y \leq \frac12)$.
\vspace{10cm}



\item Find $\pdfX$ and $\pdfY$.
\vspace{6cm}

\end{enumerate}

\end{example}

%****************************************************************************************
\newpage

\begin{example}\ %\newline
Let $\pdfXY = 2 e^{-(x+y)}$, for $0 \leq x \leq y$.

\vspace{.5cm}

\begin{enumerate}
\item Find $\pdfX$ and $\pdfY$.

\vspace{10cm}



\item Find $\Pbb(Y < 3)$.
\vspace{6cm}

\end{enumerate}


\end{example}
%----------------------------------------------------------------------------


%****************************************************************************************
\newpage
% Exercise 25.14 with different probability
\begin{example}\label{joint_absolute_diff}\ %\newline   
Let $X$ and $Y$ have constant density on the square $0 \leq X \leq 4, 0 \leq Y \leq 4$.

\vspace{.5cm}

\begin{enumerate}
\item Find $\Pbb(|X-Y| < 2)$.

%\vspace{10cm}

%****************************************************************************************
\newpage
\textit{Example \ref{joint_absolute_diff} continued.}

\item Let $M = \max(X,Y)$. Find the pdf for $M$, that is $f_M(m)$.
\vspace{10cm}


\item Let $Z = \min(X,Y)$. Find the pdf for $Z$, that is $f_Z(z)$.
%\vspace{6cm}

\end{enumerate}


\end{example}
%----------------------------------------------------------------------------




%****************************************************************************************
\newpage

\begin{example}\label{XY_pdf}\ %\newline   
Let $X$ and $Y$ have joint density $\pdfXY = \frac85(x+y)$ in the region $0 < x < 1,\ \frac12 < y <1$.
Find the pdf of the r.v. $Z$, where $Z=XY$.

\end{example}


%****************************************************************************************
\newpage
\textit{Example \ref{XY_pdf} solution continued.}

%----------------------------------------------------------------------------

%----------------------------------------------------------------------------
}  % end large font
%----------------------------------------------------------------------------



\end{document}

