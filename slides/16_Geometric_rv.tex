%\documentclass[12pt,letterpaper]{article}
\documentclass[12pt]{amsart}

\usepackage[left=1in, right=1in, top=.5in, bottom=.5in]{geometry}
\usepackage{latexsym,amssymb,amsmath,amsthm,amsopn,verbatim, mathpazo, graphicx, lmodern}
%\usepackage{latexsym,amssymb,amsmath,amsthm,amsopn,verbatim, mathpazo, graphicx}
\newcommand{\vs}{\vskip.5cm}
\newcommand{\hs}{\hskip1cm}
\newcommand{\ds}{\displaystyle}
\setlength{\parindent}{0pt}

\usepackage{hyperref}  % links, urls
\urlstyle{same}

\usepackage[T1]{fontenc}
\usepackage{libertine}
\renewcommand*\familydefault{\sfdefault}  %% Only if the base font of the document is to be sans serif


\newtheorem{theorem}{Theorem}[section]
\newtheorem{corollary}{Corollary}[theorem]
\newtheorem{lemma}[theorem]{Lemma}
\newtheorem{definition}[theorem]{Definition}
\newtheorem{example}[theorem]{Example}

\newcommand\indep{\protect\mathpalette{\protect\independenT}{\perp}}
\def\independenT#1#2{\mathrel{\rlap{$#1#2$}\mkern2mu{#1#2}}}

\usepackage{fancyhdr}
\pagestyle{fancy}
\fancypagestyle{plain}{}
%\fancyhead{} % clear all header fields
%\fancyhf{}
\rhead{BSTA 550 Ch 16}   % from fancyhdr package
\renewcommand{\headrulewidth}{1pt}
\renewcommand{\footrulewidth}{0pt}



\begin{document}


%----------------------------------------------------------------------------
\setcounter{section}{16}
{\huge  
\section*{Chapter 16: Geometric r.v.'s}
}
%----------------------------------------------------------------------------

%----------------------------------------------------------------------------
{\large 
%----------------------------------------------------------------------------

%----------------------------------------------------------------------------
%\vspace{18cm}
%\hrule
%\vspace{1cm}



\vspace{.5cm}

\textbf{Scenario:} There are repeated independent trials, each resulting in a success or failure, with constant probability of success for each trial. We are counting the number of trials until the first success.

%----------------------------------------------------------------------------
\vspace{.5cm}
\hrule
\vspace{.5cm}


%----------------------------------------------------------------------------
\begin{example}\ %\newline
We throw darts at a dartboard until we hit the bullseye. Assume throws are independent and the probability of hitting the bullseye is 0.01 for each throw. 
\begin{enumerate}
\item What is the pmf for the number of throws until we hit the bullseye?

\textbf{Solution:}
%----------------------------------------------------------------------------
\vspace{11cm}
%\hrule
%\vspace{.5cm}

\item What are the mean and variance for the number of throws until we hit the bullseye?

\textbf{Solution:}
%----------------------------------------------------------------------------
\vspace{5cm}
%\hrule
%\vspace{.5cm}



%****************************************************************************************
\newpage

\item Find the probability that our first bullseye: 
	\begin{enumerate}
	\item is on the fourth try
	
	\textbf{Solution:}
%----------------------------------------------------------------------------
\vspace{3cm}
%\hrule
%\vspace{.5cm}

	\item is on one of the first four tries
	
		\textbf{Solution:}
%----------------------------------------------------------------------------
\vspace{3cm}
	\item is after the fifth try
	
		\textbf{Solution:}



%****************************************************************************************
\newpage


%----------------------------------------------------------------------------
\vspace{5cm}
	\item is on one of the first fifty tries
		
		\textbf{Solution:}
%----------------------------------------------------------------------------
\vspace{7cm}

	\item is after the $50^{th}$ try, given that it did not happen on the first 20 tries.
			
		\textbf{Solution:}
	\end{enumerate}

%****************************************************************************************
\newpage

\item Find the expected number of misses until we hit the bullseye.
		
		\textbf{Solution:}
\end{enumerate}

\end{example}

%----------------------------------------------------------------------------
%\vspace{5cm}
%\hrule
%\vspace{.5cm}




%----------------------------------------------------------------------------
%\vspace{8cm}
%\hrule
%\vspace{.5cm}

%\textbf{Properties of Geometric r.v.'s}
%\begin{center}
% \begin{tabular}{| c | c | c | c | c | c | c} 
% \hline
%X  & 1 \\ [0.5ex] 
% \hline
%Parameter & $N$ = number of possible outcomes \\ [0.5ex] 
% \hline
%pmf & $p_X(x) = \frac{1}{N}$ \\ [0.5ex] 
% \hline
%\end{tabular}
%\end{center}



%****************************************************************************************
\newpage

%----------------------------------------------------------------------------
}  % end large font
%----------------------------------------------------------------------------



\end{document}

