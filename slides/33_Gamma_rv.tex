%\documentclass[12pt,letterpaper]{article}
\documentclass[12pt]{amsart}

\usepackage[left=1in, right=1in, top=.5in, bottom=.5in]{geometry}
\usepackage{latexsym,amssymb,amsmath,amsthm,amsopn,verbatim, mathpazo, graphicx, lmodern}
%\usepackage{latexsym,amssymb,amsmath,amsthm,amsopn,verbatim, mathpazo, graphicx}



\newcommand{\vs}{\vskip.5cm}
\newcommand{\hs}{\hskip1cm}
\newcommand{\ds}{\displaystyle}
\setlength{\parindent}{0pt}

\usepackage{hyperref}  % links, urls
\urlstyle{same}

\usepackage[T1]{fontenc}
\usepackage{libertine}
\renewcommand*\familydefault{\sfdefault}  %% Only if the base font of the document is to be sans serif


\newtheorem{theorem}{Theorem}[section]
\newtheorem{corollary}{Corollary}[theorem]
\newtheorem{lemma}[theorem]{Lemma}
\newtheorem{definition}[theorem]{Definition}
\newtheorem{example}[theorem]{Example}

\newcommand\indep{\protect\mathpalette{\protect\independenT}{\perp}}
\def\independenT#1#2{\mathrel{\rlap{$#1#2$}\mkern2mu{#1#2}}}

\newcommand\Pbb{\mathbb{P}}
\newcommand\Ebb{\mathbb{E}}
\newcommand\gl{\lambda}
\newcommand\ga{\alpha}
\newcommand\gb{\beta}
\newcommand\pdfX{f_X(x)}
\newcommand\pdfY{f_Y(y)}
\newcommand\pdfXY{f_{X,Y}(x,y)}
\newcommand\cdfX{F_X(x)}
\newcommand\cdfY{F_Y(y)}
\newcommand\cdfXY{F_{X,Y}(x,y)}
\newcommand\pdfXgY{f_{X|Y}(x|y)}
\newcommand\pdfYgX{f_{Y|X}(y|x)}

\newcommand\intd{\displaystyle\int}
\newcommand\intinft{\int_{-\infty}^{\infty}}

\usepackage{fancyhdr}
\pagestyle{fancy}
\fancypagestyle{plain}{}
%\fancyhead{} % clear all header fields
%\fancyhf{}
\rhead{BSTA 550 Ch 33}   % from fancyhdr package
\renewcommand{\headrulewidth}{1pt}
\renewcommand{\footrulewidth}{0pt}



\begin{document}


%----------------------------------------------------------------------------
\setcounter{section}{33}
{\huge  
\section*{Chapter 33: Gamma Random Variables}
}
%----------------------------------------------------------------------------

%----------------------------------------------------------------------------
{\large % begin large font
%----------------------------------------------------------------------------

%----------------------------------------------------------------------------
%\vspace{18cm}
%\hrule
%\vspace{1cm}



\vspace{.5cm}

\textbf{Scenario:} Modeling the time until the $r^{th}$ event.

\vspace{.5cm}
\hrule
\vspace{.5cm}


\textbf{Properties of gamma r.v.'s}


%****************************************************************************************
\newpage

\textbf{Remarks} 

\begin{enumerate}
\item The parameter $r$ in a Gamma($r$,$\gl$) distribution doesn't have to be a positive integer ($\mathbb{Z}^{+}$).

\item When $r\in \mathbb{Z}^{+}$, the distribution is sometimes called an Erlang($r$,$\gl$) distribution.

\item When $r$ is any positive real number, we have a general gamma distribution that is usually instead parameterized by $\ga >0 $ and $\gb >0 $, where:

	\begin{enumerate}
	\item[$\ga$] = shape parameter
	\item[$\gb$] = scale parameter
	\end{enumerate}
\end{enumerate}
%----------------------------------------------------------------------------

%----------------------------------------------------------------------------
}  % end large font
%----------------------------------------------------------------------------



\end{document}

