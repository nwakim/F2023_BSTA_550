%\documentclass[12pt,letterpaper]{article}
\documentclass[12pt]{amsart}

\usepackage[left=1in, right=1in, top=.5in, bottom=.5in]{geometry}
\usepackage{latexsym,amssymb,amsmath,amsthm,amsopn,verbatim, mathpazo, graphicx, lmodern}
%\usepackage{latexsym,amssymb,amsmath,amsthm,amsopn,verbatim, mathpazo, graphicx}
\newcommand{\vs}{\vskip.5cm}
\newcommand{\hs}{\hskip1cm}
\newcommand{\ds}{\displaystyle}
\setlength{\parindent}{0pt}

\usepackage{hyperref}  % links, urls
\urlstyle{same}

\usepackage[T1]{fontenc}
\usepackage{libertine}
\renewcommand*\familydefault{\sfdefault}  %% Only if the base font of the document is to be sans serif


\newtheorem{theorem}{Theorem}[section]
\newtheorem{corollary}{Corollary}[theorem]
\newtheorem{lemma}[theorem]{Lemma}
\newtheorem{definition}[theorem]{Definition}
\newtheorem{example}[theorem]{Example}

\newcommand\indep{\protect\mathpalette{\protect\independenT}{\perp}}
\def\independenT#1#2{\mathrel{\rlap{$#1#2$}\mkern2mu{#1#2}}}

\usepackage{fancyhdr}
\pagestyle{fancy}
\fancypagestyle{plain}{}
%\fancyhead{} % clear all header fields
%\fancyhf{}
\rhead{BSTA 550 Ch 9 }   % from fancyhdr package
\renewcommand{\headrulewidth}{1pt}
\renewcommand{\footrulewidth}{0pt}



\begin{document}


%----------------------------------------------------------------------------
\setcounter{section}{9}
{\huge  
\section*{Chapter 9: Independence and Conditioning - or, Joint Distributions}
}
%----------------------------------------------------------------------------

%----------------------------------------------------------------------------
{\large 
%----------------------------------------------------------------------------

%----------------------------------------------------------------------------
%\vspace{18cm}
%\hrule
\vspace{1cm}

%----------------------------------------------------------------------------
\begin{definition}
The \textbf{joint pmf} of a pair of discrete r.v.'s $X$ and $Y$ is 
$$
p_{X,Y}(x,y) = \mathbb{P}(X=x\ and\ Y=y) = \mathbb{P}(X=x, Y=y)
$$ 
\end{definition}

%----------------------------------------------------------------------------

\vspace{1cm}

%----------------------------------------------------------------------------
\begin{example}\label{DrawsBox}
Let $X$ and $Y$ be two random draws from a box containing balls labelled 1, 2, and 3 without replacement. 

\begin{enumerate}
\item Find $p_{X,Y}(x,y)$.
\item Find $\mathbb{P}(X+Y=3).$
\item Find $\mathbb{P}(Y = 1).$
\item Find $\mathbb{P}(Y \leq 2).$
\end{enumerate}

\end{example}

\textbf{Solution:}


%----------------------------------------------------------------------------
%\vspace{18cm}
%\hrule
%\vspace{.5cm}





%****************************************************************************************
\newpage

\textbf{Remarks:} Some properties of joint pmf's:%\newline

\begin{itemize}
\item A joint pmf $p_{X,Y}(x,y)$ must satisfy the following properties:
	\begin{itemize}
	\item $p_{X,Y}(x,y)\geq 0$ for all $x, y$.
	\item $\sum \limits_{\{all\ x\}} \sum \limits_{\{all\ y\}} p_{X,Y}(x,y)=1$.
	\end{itemize}
\vspace{1cm}
\item Marginal pmf's:
	\begin{itemize}
	\item $p_X(x) = \sum \limits_{\{all\ y\}} p_{X,Y}(x,y)$
	\item $p_Y(y) = \sum \limits_{\{all\ x\}} p_{X,Y}(x,y)$	
	\end{itemize}
\end{itemize}


%----------------------------------------------------------------------------
%\vspace{1cm}
%\hrule
%\vspace{.5cm}

%----------------------------------------------------------------------------

%****************************************************************************************
\newpage

%----------------------------------------------------------------------------
\begin{definition}
The \textbf{joint cdf} of a pair of discrete r.v.'s $X$ and $Y$ is 
$$
F_{X,Y}(x,y) = \mathbb{P}(X \leq x\ and\ Y \leq y) = \mathbb{P}(X \leq x, Y \leq y)
$$ 
\end{definition}

%----------------------------------------------------------------------------

%----------------------------------------------------------------------------
\vspace{.5cm}
\hrule
\vspace{.5cm}

%----------------------------------------------------------------------------


%----------------------------------------------------------------------------
\begin{example}\label{DrawsBoxCDF}
Find the joint cdf $F_{X,Y}(x,y)$ for the joint pmf $p_{X,Y}(x,y)$ in Example \ref{DrawsBox}.

\end{example}

\textbf{Solution:}




%****************************************************************************************
\newpage
%----------------------------------------------------------------------------
\begin{example}
Find the marginal cdfs $F_{X}(x)$ and $F_{Y}(y)$ for Example \ref{DrawsBoxCDF}.

\end{example}

\textbf{Solution:}


%----------------------------------------------------------------------------

%----------------------------------------------------------------------------
\vspace{14cm}
\hrule
\vspace{.5cm}

%----------------------------------------------------------------------------
\textbf{Remark:} Some properties of joint cdf's:%\newline



%----------------------------------------------------------------------------
%\vspace{1cm}
%\hrule
%\vspace{.5cm}

%----------------------------------------------------------------------------


%****************************************************************************************
\newpage

\textbf{Independence and Conditioning}
\vspace{.5cm}

Recall that for \textit{events} $A$ and $B$,
\begin{itemize}
\item $\mathbb{P}(A|B) = \frac{\mathbb{P}(A \cap B)}{\mathbb{P}(B)}$
\item $A$ and $B$ are independent if and only if 
	\begin{itemize}
	\item $\mathbb{P}(A|B) = \mathbb{P}(A)$ 
	\item$\mathbb{P}(A \cap B) = \mathbb{P}(A)\cdot\mathbb{P}(B)$
	\end{itemize}
\end{itemize}

\vspace{1cm}

Independence and conditioning are defined similarly for r.v.'s, since% \newline
 $$p_X(x) = \mathbb{P}(X=x)\ \mathrm{and}\ \ p_{X,Y}(x,y) = \mathbb{P}(X = x ,Y = y).$$

\vspace{1cm}
%----------------------------------------------------------------------------
\begin{definition}
The \textbf{conditional pmf} of a pair of discrete r.v.'s $X$ and $Y$ is defined as
$$
p_{X|Y}(x|y) = \mathbb{P}(X = x |Y = y) = \frac{\mathbb{P}(X = x\ and\ Y = y)}{\mathbb{P}(Y = y)}
=\frac{p_{X,Y}(x,y) }{p_{Y}(y) }
$$ 
if $p_{Y}(y)  > 0$.
\end{definition}

%----------------------------------------------------------------------------

%----------------------------------------------------------------------------
\vspace{1cm}
%\hrule
%\vspace{.5cm}

%----------------------------------------------------------------------------


\textbf{Remark:} The following properties follow from the conditional pmf definition:%\newline




%****************************************************************************************
\newpage


%----------------------------------------------------------------------------
\begin{example}
Using $X$ and $Y$ from Example \ref{DrawsBox}:
\begin{enumerate}
\item Find $p_{X|Y}(x|y)$.
\item Are $X$ and $Y$ independent? Why or why not?
\end{enumerate}


\end{example}

\textbf{Solution:}


%----------------------------------------------------------------------------
\vspace{18cm}
\hrule
\vspace{.5cm}

%----------------------------------------------------------------------------


\textbf{Remark:} 
\begin{itemize}
\item To show that $X$ and $Y$ are \textit{not} independent, we just need to find one counterexample. 
\item However, to show that they are independent, we need to verify this for all possible pairs of $x$ and $y$.
\end{itemize}




%****************************************************************************************
\newpage


%----------------------------------------------------------------------------
\begin{example}\label{Die4sided} \textbf{Hypothetical 4-sided die}
\begin{itemize} 
\item Suppose you have a 4-sided die, and you roll the 4-sided die until the first 4 appears.
\item Let $X$ be the number of rolls required until (and including) the first 4.
\item After the first 4, you keep rolling it again until you roll a 3. 
\item Let $Y$ be the number of rolls, after the first 4, required until (and including) the 3.
\end{itemize}

\begin{enumerate}
\item Find $p_{X,Y}(x,y)$.
\item Using $p_{X,Y}(x,y)$, find $p_{Y}(y)$.
\item Find $p_{X}(x)$.
\item Are $X$ and $Y$ are independent? Why or why not?
\item Find $F_{X,Y}(x,y)$.
\end{enumerate}


\end{example}

\textbf{Solution:}

%****************************************************************************************
\newpage
    
Example \ref{Die4sided} cont'd.    
%----------------------------------------------------------------------------
%\vspace{20cm}
%\hrule
%\vspace{.5cm}

%----------------------------------------------------------------------------



%----------------------------------------------------------------------------



%----------------------------------------------------------------------------
}  % end large font
%----------------------------------------------------------------------------



\end{document}

